% !TeX root = ../main.tex
\chapter{面向场景的评估系统实现}

在前文完成系统总体架构设计与核心模块功能分析的基础上,
本章将介绍面向场景的静态分析工具评估系统的具体实现过程。

本章基于场景知识库的设计,
围绕“场景生成—场景管理—评估计算”这一核心工程流程,
详细阐述系统关键功能模块的实现方法与工程细节,
首先介绍场景知识库的工程实现,
然后重点说明在引入大语言模型后,
系统如何在不确定性条件下尽量保证结果的质量;
随后说明场景资产在系统中的管理、去重与分类实现;
接着重点描述评估流程的自动化实现方式;
最后从工程角度分析系统在工具替换与功能扩展方面的实现特性。

\iffalse
	\chapter{面向场景的评估系统实现}

	\section{场景知识库与数据模型设计}
	\subsection{场景实体与核心字段设计}
	\subsection{场景空间维度的数据承载方式}
	\subsection{自然语言描述与结构化信息的映射}

	\section{场景生成机制实现}
	\subsection{基于自然语言的问题抽象}
	\subsection{正向梳理的场景生成流程}
	\subsection{被动补充的场景生成流程}
	\subsection{大语言模型不确定性的工程约束}

	\section{场景去重、分类与管理实现}
	\subsection{场景等价性与去重策略}
	\subsection{基于场景空间的分类实现}
	\subsection{场景生命周期管理}

	\section{评估流程自动化实现}
	\subsection{评估流程总体调度机制}
	\subsection{工具执行与结果采集}
	\subsection{覆盖度指标计算与结果归一化}

	\section{系统可扩展性与替换性分析}
	\subsection{分析工具的替换机制}
	\subsection{场景维度的扩展能力}
	\subsection{评估流程的模块解耦}
\fi

\section{场景知识库的实现}

场景知识库需要支持长期演化,
同时服务于上层三个模块的功能,
需要同时作为RAG的检索库和测试集库。
本节首先介绍场景实体的核心字段,
再介绍场景空间的信息如何由字段数据承载,
最后介绍如何通过自然语言将结构化信息和代码结合到一起。

\subsection{场景等实体的核心字段}

根据本文设计的ER图\ref{fig:ER图},
场景实体是整个知识库的核心单元,
一个场景实体需要记录代码、
空间信息、
属于哪个缺陷等信息,
表\ref{tab:场景实体核心字段}逐一介绍了场景实体的核心字段。

\begin{table}
	\centering
	\caption{场景实体核心字段}
	\label{tab:场景实体核心字段}
	\begin{tabular}{ccp{9cm}}
		\hline
		字段                   & 格式     & 含义                       \\
		\hline
		description          & string & 代码内容的自然语言描述              \\
		good                 & dict   & good代码内容,键值为file:content \\
		bad                  & dict   & bad代码内容,键值为file:content  \\
		weakness             & int    & 所属的缺陷CWE ID              \\
		Program\_Span        & int    & 场景空间的 $P$ 维度:程序跨度        \\
		Semantic\_Complexity & int    & 场景空间的 $S$ 维度:语义建模复杂度     \\
		Path\_Complexity     & int    & 场景空间的 $C$ 维度:路径结构复杂度     \\
		Path\_Depth          & int    & 场景空间的 $L$ 维度:路径深度与状态空间控制 \\
		Reachability         & int    & 场景空间的 $R$ 维度:有无不可达路径     \\
		\hline
	\end{tabular}
\end{table}

其中description、good、bad三者是场景基础信息,
其他的是场景空间信息。

在知识库中除了场景实体作为核心表,
还有覆盖表和应该覆盖表,
都是一个多对多的关系,
分别记录运行之后规则和场景的覆盖结果、
规则和缺陷之间应该覆盖的真实值。

为了满足系统大量的检索需求,
场景知识库使用ElasticSearch(ES)承载,
ES的检索功能包括向量检索和其他多种混合检索,
非常适合本文系统的需求。

\subsection{场景空间信息的数据承载方式}

场景在空间中的信息并不是连续的,
如章节\ref{sec:场景空间}所述,
也并不是每个场景都有所有的维度,
维度是根据缺陷语义而不同的。
根据本文所涉及的刻度,
场景利用不同字段中的枚举或数字来维护维度信息。
数字使用前缀编码的方式,
0代表最简单的情况,
也可以代表该维度是平凡的;
第一个非零数字表示刻度,
后面的数字表示刻度内部的子维度,
维度信息刻画如表\ref{tab:场景空间信息字段含义},
这些字段体现的是本文目前的刻度设计,
前缀编码的方式也适合后续拓展。

\begin{table}
	\centering
	\caption{场景空间信息字段含义}
	\label{tab:场景空间信息字段含义}
	\begin{tabular}{lp{9cm}}
		\hline
		字段       & 含义                \\
		\hline
		weakness & 所属的缺陷CWE ID       \\
		\hline

		\makecell[l]{
		Program\_Span                \\
		程序跨度                         \\
		}
		         & \makecell[l]{
		0:单函数内部;                     \\
		1:上下文无关跨函数;                  \\
		2:上下文敏感跨函数;                  \\
		3x:跨文件,后缀 x 表示跨文件层数;         \\
		9:全程序(本维度上限)                 \\
		}
		\\
		\hline

		\makecell[l]{
		Semantic\_Complexity         \\
		语义建模复杂度                      \\
		}
		         & \makecell[l]{
		0:无语义;                       \\
		1:混淆正则匹配的语义,需要理解语法;          \\
		2:混淆if、while等\tc~语法,需要理解控制流; \\
		3:混淆控制流的语义,需要理解状态;           \\
		}
		\\
		\hline

		\makecell[l]{
		Path\_Complexity             \\
		路径结构复杂度                      \\
		}
		         & \makecell[l]{
		0:顺序结构;                      \\
		11:在if路径中;                   \\
		12:在else路径中;                 \\
		21:在while的路径中;               \\
		22:在for的路径中;                 \\
		23:在while的break路径中;          \\
		24:在for的break路径中;            \\
		3:在if和while/for复合的路径中;       \\
		9:在无限复杂的路径中(该维度的上限)          \\
		}
		\\
		\hline

		\makecell[l]{
		Path\_Depth                  \\
		路径深度与状态空间控制                  \\
		}
		         & \makecell[l]{
		0:不需要控制路径深度;                 \\
		1:if/else需要合并;               \\
		2:switch中多个case需要合并;         \\
		3:有无限增长的路径,工具规则需要能够兜底;       \\
		4x:在非平凡循环之外的if语句中,           \\
		后缀x表示非平凡循环层数,                \\
		需要有一定的展开能力;                  \\
		5:在大循环的末端,                   \\
		需要使用widen技术划分循环;             \\
		}
		\\
		\hline

		\makecell[l]{
		Reachability                 \\
		有无不可达路径                      \\
		}
		         & \makecell[l]{
		0:无不可达路径;                    \\
		1:有不可达路径;                    \\
		}
		\\
		\hline
	\end{tabular}
\end{table}

举例来说,
对于表达式\ref{eq:场景1}
描述的一个空间信息,

\begin{equation}
	\mathcal{p} = \langle
	D = 476,
	P = 1,
	S = 3,
	C = 12,
	L = 45,
	R = 0
	\rangle
	\label{eq:场景1}
\end{equation}

它是比较复杂的场景,
除了维度R都是非平凡的,
其含义是:
一个缺陷\texttt{CWE-476:NULL\_Pointer\_Dereference},
发生在如下的条件中:
数据的源和触发发生在上下文无关的跨函数之间;
触发条件发生在状态某种特定状态下;
发生在else分支中;
发生在非平凡循环外的分支中,
循环深度为5;
没有不可达路径。
具体代码见附录\ref{lst:场景1实际代码}


\subsection{自然语言描述与结构化信息的映射}
todo
场景需要被检索,
但是代码检索是另一个比较广泛的课题,
不是本文的重点。
本问需要的是通过自然语言检索到相关的代码片段,
生成功能分为两部分,
正向梳理和被动补充,
这两个方向的源头都是被描述成自然语言的代码问题。
下面分别举例正向梳理和被动补充时复杂空间信息,


从这个场景空间信息到场景的检索是不直接的,
需要将场景描述为自然语言,
以便场景空间信息和代码内容之间互相理解。




\subsection{评估结果与覆盖关系的存储方式}

\section{场景生成机制实现}
\section{场景去重、分类与管理实现}
\section{评估流程自动化实现}
\section{系统可扩展性与替换性分析}