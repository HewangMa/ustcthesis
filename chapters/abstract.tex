% !TeX root = ../main.tex

\ustcsetup{
  keywords  = {C/C++静态分析工具,分析精度,缺陷场景,检索增强生成},
  keywords* = {C/C++ static analysis tools, analysis accuracy, defect scenarios , retrieval enhancement generation },
}

\begin{abstract}
  % 摘要是论文内容的总结概括,应简要说明论文的研究目的、基本研究内容、研究方法或过程、结果和结论,突出论文的创新之处。
  % 摘要应具有独立性和自明性,即不用阅读全文,就能获得论文必要的信息。
  % 摘要中不宜使用公式、图表,不引用文献。

  % 摘要分中文和英文两种,中文在前,英文在后,内容及段落须相互呼应。博士论文中文摘要一般800~1000个汉字,硕士论文中文摘要一般600个汉字。
  % 英文摘要的篇幅参照中文摘要。

  % 论文的关键词,是为了文献标引工作从论文中选取出来用以表示全文主题内容信息的单词或术语。建议关键词数量不超过8个,每个关键词之间用分号间隔。

  % 英文摘要部分的标题为“ABSTRACT”。
  % 每个关键词第一个字母大写,关键词之间用半角逗号加空一格间隔,英文关键词与中文关键词须相互呼应。

  本文针对企业在用C/C++静态分析工具在复杂代码场景中覆盖能力评估不足的问题,设计并实现一种面向具体规则、基于场景的静态分析工具分析精度评估系统。为解决传统评估方式依赖人工构造场景、覆盖面不足、难以量化等问题,本文深入研究了这类工具特性、关键的评估要点,设计引入“场景工厂”与“场景资产管理”两大核心模块,结合检索增强生成驱动的场景生成技术,构建具备演化能力的场景知识库,来实现工具分析精度的动态评估。

  本文的场景工厂采用检索增强生成和责任链设计模式,结合企业真实问题与通用静态分析规则,构造跨文件、深调用链等复杂代码场景,用于模拟静态分析器可能面临的挑战性代码结构;本文的场景资产管理支持对场景进行增删改查、分类和去重,该模块还包含量化计算方法,提供在不同规则维度下的精准度、召回率、F1 分数等指标计算功能;本文的场景知识库是系统的核心数据库,该数据库根据系统需求设计,提供评估所需关键信息,并支撑场景的使用和维护。 

  考虑到本文目标系统实现主要针对企业内部专用的多种C/C++静态分析工具,所以,本文在测试验证环节,选用开源的clang-tidy  工具进行可验证实验,以展示本文的评估方法及所实现的目标系统的适用性、通用性和可用性。
  
  本文目标系统的应用,可以为各类面向C/C++语言的静态分析工具提供自动化评估测评功能,从而为这类静态分析工具的改进与优化提供决策依据。

\end{abstract}

\begin{abstract*}
  The length of the English abstract should refer to that of the Chinese abstract.
  The title of the English abstract is “ABSTRACT”.
  The first letter of each keyword should be capitalized, and keywords should be separated by a halfwidth comma and a space.
  The English keywords and Chinese keywords should correspond to each other.
\end{abstract*}
