% !TeX root = ../main.tex

\ustcsetup{
	keywords  = {C/C++静态分析工具,分析精度,缺陷场景,检索增强生成},
	keywords* = {C/C++ static analysis tools, analysis accuracy, defect scenarios , retrieval enhancement generation },
}

\begin{abstract}
	%% 开题报告版本

	% 本文针对企业在用C/C++静态分析工具在复杂代码场景中覆盖能力评估不足的问题,设计并实现一种面向具体规则、基于场景的静态分析工具分析精度评估系统。为解决传统评估方式依赖人工构造场景、覆盖面不足、难以量化等问题,本文深入研究了这类工具特性、关键的评估要点,设计引入“场景工厂”与“场景资产管理”两大核心模块,结合检索增强生成驱动的场景生成技术,构建具备演化能力的场景知识库,来实现工具分析精度的动态评估。

	% 本文的场景工厂采用检索增强生成和责任链设计模式,结合企业真实问题与通用静态分析规则,构造跨文件、深调用链等复杂代码场景,用于模拟静态分析器可能面临的挑战性代码结构;本文的场景资产管理支持对场景进行增删改查、分类和去重,该模块还包含量化计算方法,提供在不同规则维度下的精准度、召回率、F1 分数等指标计算功能;本文的场景知识库是系统的核心数据库,该数据库根据系统需求设计,提供评估所需关键信息,并支撑场景的使用和维护。

	% 考虑到本文目标系统实现主要针对企业内部专用的多种C/C++静态分析工具,所以,本文在测试验证环节,选用开源的clang-tidy  工具进行可验证实验,以展示本文的评估方法及所实现的目标系统的适用性、通用性和可用性。

	% 本文目标系统的应用,可以为各类面向C/C++语言的静态分析工具提供自动化评估测评功能,从而为这类静态分析工具的改进与优化提供决策依据。



	%% 导师版本 %% 

	% 企业在复杂代码场景中使用C/C++ 静态分析工具时,存在依赖人工构造场景、覆盖面不足、难以量化等问题。为解决传统评估、评测方式覆盖面及评估能力不足问题,本文深入研究了主流的C/C++ 静态分析工具的技术特性、评估的关键技术,设计引入了“场景工厂”与“场景资产管理”两大核心模块,结合检索增强生成驱动的场景生成技术,构建具备演化能力的场景知识库,以面向动态可变规则的方式,来实现C/C++静态分析工具的精度动态评估系统。这对于为这类静态分析工具的改进与优化提供更精准的决策依据具有重要意义。

	% 本文的主要工作和贡献包括:
	% \begin{enumerate}
	% 	\item 基于检索增强生成驱动的场景生成技术,构建具备演化能力的场景知识库,并基于该知识库,设计并实现了一种基于场景的静态分析工具分析精度评估系统。
	% 	\item 本文的场景工厂采用检索增强生成和责任链设计模式,结合企业真实问题与通用静态分析规则,构造跨文件、深调用链等复杂代码场景,用于模拟静态分析器可能面临的挑战性代码结构;其中,场景资产管理支持对场景进行增删改查、分类和去重,该模块还包含量化计算方法,提供在不同规则维度下的精准度、召回率、F1 分数等指标计算功能。而作为系统的核心数据库的场景知识库管理模块,可根据企业需求设计并支撑场景的使用和维护,为评估提供所需关键且充分的用例场景。
	% 	\item 考虑到本文目标系统实现主要针对企业内部专用的多种 C/C++ 静态分析工具,所以,本文在测试验证环节,选用开源的 clang-tidy 工具进行可验证实验,以展示本文的评估方法及所实现的目标系统的适用性、通用性和可用性。
	% \end{enumerate}

	% 本文目标系统的现已开发完成,并在企业中投入了应用。应用结果表明,该系统可以为各类面向 C/C++ 语言的静态分析工具提供更为精准有效的自动化评估测评功能。

	\iffalse
		本文针对企业在复杂代码场景中使用 C/C++ 静态分析工具时,
		普遍存在依赖人工构造测试场景、
		覆盖能力不足且评估结果难以量化的问题,
		提出了一种面向具体规则、
		基于缺陷场景的静态分析工具覆盖能力评估方法,
		并设计实现了相应的自动化评估系统。

		为突破传统评估方式以静态用例为核心、
		难以反映分析能力边界的局限,
		本文从静态分析工具的通用分析假设与能力约束出发,
		将工具覆盖能力建模为其在一组结构化缺陷场景空间中的可检测性问题。
		围绕该建模思路,
		本文设计了一种具备演化能力的场景知识库,
		并引入检索增强生成技术,
		以支持复杂代码场景的自动构造与持续扩展,
		从而实现对静态分析工具分析精度的动态评估。

		在系统实现层面,
		本文设计并实现了包含场景生成、
		场景管理与评估计算等功能模块的评估系统原型。
		该系统能够面向不同静态分析规则,
		自动构造跨文件、深调用链等具有代表性的复杂缺陷场景,
		并在此基础上对工具的精准度、
		召回率及 F1 指标进行量化计算。

		考虑到本文研究对象主要面向企业内部使用的多种 C/C++ 静态分析工具,
		本文选取开源工具 clang-tidy 作为实验对象,
		对所提出的评估方法和系统进行了可验证实验。
		实验结果表明,
		该方法能够有效刻画静态分析工具在不同规则维度下的覆盖能力特征,
		具备良好的适用性和通用性。

		本文的研究成果已在企业环境中得到实际应用,
		可为 C/C++ 静态分析工具的评估、
		选型及优化提供更加客观、可量化的决策依据。
	\fi
	todo
\end{abstract}

\begin{abstract*}
	todo
\end{abstract*}
