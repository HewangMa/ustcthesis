% !TeX root = ../main.tex

% Experiments
% → 证明你的方法不是偶然有效
\chapter{实验与分析}
\label{ch:实验与分析}

为了保证场景知识库服务的可重现性、
环境隔离性与易运维性,
本文采用Docker容器化方式在本地/服务器上部署Elasticsearch(以下简称ES)与Kibana。
使用Docker的主要理由包括:
消除宿主环境依赖差异,
保证不同实验/评测环境的一致性;
便于版本固定与回滚,
利于科研复现;
简化运维与资源隔离,
便于在CI或实验平台中自动化启动与销毁实例;
通过卷(volume)实现数据持久化,
便于长期保存索引与检索结果,
也就是容器和数据的隔离。
Docker技术使用docker-compose部署多个容器,
本文使用的docker-compose部署示例如配置\ref{lst:docker-compose}所示。


\section{实验设计与评估目标}
\section{不同工具规则的表现分析}
\section{与传统评估方式的对比}
\section{威胁分析与讨论}

% \subsection{二级节标题}

% \subsubsection{三级节标题}

% \section{脚注}

% Lorem ipsum dolor sit amet, consectetur adipiscing elit, sed do eiusmod tempor
% incididunt ut labore et dolore magna aliqua.
% \footnote{Ut enim ad minim veniam, quis nostrud exercitation ullamco laboris
% 	nisi ut aliquip ex ea commodo consequat.
% 	Duis aute irure dolor in reprehenderit in voluptate velit esse cillum dolore
% 	eu fugiat nulla pariatur.}
