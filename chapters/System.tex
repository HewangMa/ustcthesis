% !TeX root = ../main.tex
\chapter{面向场景的评估系统设计}

本章在章节\ref{sec:覆盖能力评估问题建模}的基础上,
给出一个可工程化实现、可扩展、可复用的系统设计方案。

\section{系统需求分析与设计目标}

\subsection{功能需求分析}

通过章节\ref{sec:覆盖能力评估问题建模}对静态分析工具覆盖能力评估问题的建模,
可以将评估系统所需支撑的功能抽象为两类核心任务:
一类是围绕评估场景的生成、组织与管理;
另一类是基于既有场景与工具检查结果,
对工具能力指标进行计算与整理。
这两类任务分别对应评估系统在数据准备与评估分析中的职责划分。

围绕上述功能需求,
本文对系统的主要参与方与功能交互进行了抽象,
其功能需求用例如图 \ref{fig:需求用例图} 所示。

\begin{figure}[ht]
	\centering
	\includegraphics[width=0.5\textwidth]{需求用例图.png}
	\caption{需求用例图}
	\label{fig:需求用例图}
\end{figure}

todo:
对用户的解释

其中,“管理场景空间”用例覆盖了场景的
增加、去重、分类与删除等生命周期管理操作;
“生成场景”用例描述了系统在既有场景空间基础上产生新评估场景的能力;
“计算工具指标”用例用于基于场景与工具检查结果,
对规则及工具层面的评估指标进行计算与整理。

\subsection{功能边界}

在系统设计之前,
为避免评估系统职责膨胀,
有必要明确系统的功能边界,
本文系统的功能边界如下:

\begin{itemize}
	\item 本文的静态分析工具评估系统并不是检查运行系统,
	      不使用任何静态分析工具进行实际的检查工作,
	      不对多种工具进行集成运行,
	      而是基于检查结果进行评估指标的计算和分析。

	\item 本文的静态分析工具评估系统并不是报告解析系统,
	      对于检查结果,
	      只获取一个场景测例是否被告警这一个数据,
	      不获取帮助程序修改的报错路径等等信息。

	\item 本文的静态分析工具评估系统并不是工具能力呈现面板,
	      作为一个后端系统,
	      进行评估指标数据的计算和整理,
	      提供数据获取的接口,
	      但不提供数据的展示面板。
\end{itemize}

在以上功能边界的限制下,
本文得以聚焦于静态分析工具能力的评估,
面对静态分析工具评估这样的需求设计一个系统,
需要有更加具体的系统设计目标和原则。

\subsection{设计目标和原则}

% 可持续性评估
本文系统的第一个目标是持续性评估。
本文系统并非一次工具评估过程,
一次性给出当前工具的能力现状;
而是长期支撑工具能力认知,
在工具能力不断发展、
软件缺陷不断冒头的今天,
软件行业并不是要在几个现成的工具中做选择,
而是要更清楚的了解当前工具能力。
对于可持续性评估目标,
本文采用数据驱动的设计原则,
建立场景资产数据库,
承载场景空间,
通过持续维护场景空间和工具的检查结果,
对工具能力进行实时的评估。

% 评估结果可复现
本文系统的第二个目标是评估可复现、可验证。
工具的开发者对工具的迭代是非常迅速的,
往往修改一个工具配置就可以影响某个测试用例的检查结果。
为此,本文为了实现快速复现评估结果,
对比不同配置下的工具能力,
本文系统使用解耦数据和计算、
设计定式评估流程的设计原则,
将对场景资产的管理和基于场景的计算两类功能进行分离,
基于数据和固定的评估流程,
实现评估结果的可复现和验证。

% 可拓展性
本文系统的第三个目标是可拓展。
本文系统的组成部分涉及多种功能的组合,
为了未来评估系统的易于升级、拓展和发展,
必须采用模块化设计原则,
设计高内聚、低耦合的模块,
实现各自的功能。
场景是本文系统的核心资产,
各个模块基于场景实现功能实现和消息通讯。

上述设计目标及其对应的设计原则与工程体现关系如表 \ref{tab:本文系统的设计目标和原则} 所示。

\begin{table}
	\centering
	\caption{本文系统的设计目标和原则}
	\label{tab:本文系统的设计目标和原则}
	\begin{tabular}{p{3cm}p{5cm}p{5cm}}
		\hline
		设计目标   & 设计原则         & 工程体现        \\
		\hline
		可持续性评估 & 数据驱动         & 场景资产数据库     \\
		可复现评估  & 解耦数据和计算、定式流程 & 设计独立的指标计算模块 \\
		可拓展性   & 模块化设计        & 模块之间不互相依赖   \\
		\hline
	\end{tabular}
\end{table}

\section{系统总体架构}

本节结合本系统的外部环境、
自身的功能和设计原则,
介绍系统总体架构。

\begin{figure}[ht]
	\centering
	\includegraphics[width=0.8\textwidth]{系统架构图.png}
	\caption{系统架构图}
	\label{fig:系统架构图}
\end{figure}

如\ref{fig:系统架构图}所示,
本文系统外部面对两个系统,
集成化运行系统和可信展示平台,
前者承担着集成、配置、运行工具,
以及解析工具报告的功能;
后者承担着与测试人员互动、
展示工具指标的功能。
本文系统不直接对接测试人员,
而是面对这两个平台发出的各种消息命令。
面对可信展示平台,
本文系统接受各种场景管理命令,
提供工具指标数据;
面对集成化运行系统,
本文系统接受工具的检查报告,
发出计算请求并提供测试集。

为了满足系统的功能,
本文系统分为三个功能模块,
构建在一个数据库之上。
\textbf{场景工厂模块}负责生成场景,
为了保证场景生成质量,
聚焦在生成和检查任务上,
仅负责场景的增量动作
\textbf{场景资产管理模块}负责管理场景生命周期,
包括场景去重、分类等功能;
\textbf{工具指标计算模块}负责计算工具的两类指标,
前两个模块负责数据的治理层面,
计算模块则不对数据做任何修改,
仅基于数据进行指标计算和整理。
\textbf{场景知识库}负责承载场景空间和指标数据,
通过多个表的组织,
它不仅是测试用例的组织载体,
也是场景知识载体,
还是工具指标的检索载体;
三个功能模块并没有过程耦合,
都基于场景知识库的数据进行通信。

下面介绍本文系统各模块的设计。

\section{场景知识库}

% 回答为什么用知识库承载场景空间
根据章节\ref{sec:覆盖能力评估问题建模}对场景空间的定义解释和拓展方法,
承载场景空间的数据库要具备两方面的能力:
测试集组织和场景拓展参考,
前者是直接用于工具评估的数据,
后者是拓展场景的基础。
要同时拥有这两个能力,
就需要设计完整的结构化数据。
如果仅以传统测试用例集合的形式存储场景,
将难以表达场景之间的语义关系以及规则、
缺陷与工具之间的关联信息,
不利于场景空间的持续拓展与能力评估结果的系统化分析。

本文系统采用数据作为通信桥梁的设计原则,
在一个数据库中,
承载多种实体,
以场景空间为核心,
组织工具规则等对象,
以及各种实体之间的关系,
让实体之间的转换和功能模块相匹配。
经过分析整理,
本文将场景知识库的ER图设计如图\ref{fig:ER图}

\begin{figure}[ht]
	\centering
	\includegraphics[width=0.8\textwidth]{ER图.png}
	\caption{场景知识库ER图}
	\label{fig:ER图}
\end{figure}

对于场景知识库中关键的实体和关系的解释如表\ref{tab:场景知识库中关键的实体和关系}。

\begin{table}
	\centering
	\caption{场景知识库中关键的实体和关系}
	\label{tab:场景知识库中关键的实体和关系}
	\begin{tabular}{cc}
		\hline
		实体/关系 & 含义                      \\
		\hline
		场景    & 场景空间点的实例,是基本的测试单元       \\
		规则    & 最小单位的测试对象               \\
		工具    & 在规则的评估指标完成后将汇总计算工具的评估指标 \\
		覆盖    & 工具对场景的实际告警结果            \\
		应当覆盖  & 基于缺陷语义定义的规则期望覆盖关系       \\
		\hline
	\end{tabular}
\end{table}

\section{场景工厂模块}

本文的场景工厂负责生成场景实例,
从架构图\ref{fig:系统架构图}可以看出,
输入是从可信平台传入的抽象约束,
其中包括目标场景的约束维度和目标缺陷,
输出是可评估的结构化场景,
模块同时必须承担质量控制责任,
生成结果要符合静态分析工具检查的各种要求。
总之,
场景工厂模块的生成对象是可被评估系统理解和消费的场景实例。

从设计角度出发,
为了保证场景工厂生成的质量和工程合法性,
本文在生成之外增加了一个校验环节,
这就是 RAG 加责任链的模式。

为了实现这样的设计,
本文设计了本模块的核心类图如
% todo
% \ref{}。

其中各个核心类的

从流程上说,
为了完成生成和校验的过程,

\section{场景资产管理模块}

本文的场景资产管理模块负责

\section{工具指标计算模块}
