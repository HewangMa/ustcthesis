% !TeX root = ../main.tex
\chapter{面向场景的评估系统设计}

本章在章节\ref{sec:覆盖能力评估问题建模}的基础上,
给出一个可工程化实现、可扩展、可复用的系统设计方案。

\section{系统需求分析与设计目标}

\subsection{功能需求分析}

通过章节\ref{sec:覆盖能力评估问题建模}对静态分析工具覆盖能力评估问题的建模,
可以将评估系统所需支撑的功能抽象为两类核心任务:
一类是围绕评估场景的生成、组织与管理;
另一类是基于既有场景与工具检查结果,
对工具能力指标进行计算与整理。
这两类任务分别对应评估系统在数据准备与评估分析中的职责划分。

围绕上述功能需求,
本文对系统的主要参与方与功能交互进行了抽象,
其功能需求用例如图\ref{fig:需求用例图} 所示。

\begin{figure}[ht]
	\centering
	\includegraphics[width=0.5\textwidth]{需求用例图.png}
	\caption{需求用例图}
	\label{fig:需求用例图}
\end{figure}

其中,
用户“可信平台”是评估人员直接使用的前端平台;
用户“集成化运行平台”是工具运行、配置、报告解析的集成化平台;
用例“生成场景”描述了系统在既有场景空间基础上产生新评估场景的能力;
用例“管理场景空间”覆盖了场景的去重、分类与删除等生命周期管理操作;
用例“计算工具指标”用于基于场景与工具检查结果,
对规则及工具层面的评估指标进行计算与整理。

\subsection{功能边界}

在系统设计之前,
为避免评估系统职责膨胀,
有必要明确系统的功能边界,
本文系统的功能边界如下:

\begin{itemize}
	\item 本文的静态分析工具评估系统并不是检查运行系统,
	      不使用任何静态分析工具进行实际的检查工作,
	      不对多种工具进行集成运行,
	      而是基于检查结果进行评估指标的计算和分析。

	\item 本文的静态分析工具评估系统并不是报告解析系统,
	      对于检查结果,
	      只获取一个场景测例是否被告警这一个数据,
	      不获取帮助程序修改的报错路径等等信息。

	\item 本文的静态分析工具评估系统并不是工具能力呈现面板,
	      作为一个后端系统,
	      进行评估指标数据的计算和整理,
	      提供数据获取的接口,
	      但不提供数据的展示面板。
\end{itemize}

在以上功能边界的限制下,
本文得以聚焦于静态分析工具能力的评估,
面对静态分析工具评估这样的需求设计一个系统,
需要有更加具体的系统设计目标和原则。

\subsection{设计目标和原则}

% 可持续性评估
本文系统的第一个目标是持续性评估。
本文系统并非一次工具评估过程,
一次性给出当前工具的能力现状;
而是长期支撑工具能力认知,
在工具能力不断发展、
软件缺陷不断冒头的今天,
软件行业并不是要在几个现成的工具中做选择,
而是要更清楚的了解当前工具能力。
对于可持续性评估目标,
本文采用数据驱动的设计原则,
建立场景资产数据库,
承载场景空间,
通过持续维护场景空间和工具的检查结果,
对工具能力进行实时的评估。

% 评估结果可复现
本文系统的第二个目标是评估可复现、可验证。
工具的开发者对工具的迭代是非常迅速的,
往往修改一个工具配置就可以影响某个测试用例的检查结果。
为此,本文为了实现快速复现评估结果,
对比不同配置下的工具能力,
本文系统使用解耦数据和计算、
设计定式评估流程的设计原则,
将对场景资产的管理和基于场景的计算两类功能进行分离,
基于数据和固定的评估流程,
实现评估结果的可复现和验证。

% 可拓展性
本文系统的第三个目标是可拓展。
本文系统的组成部分涉及多种功能的组合,
为了未来评估系统的易于升级、拓展和发展,
必须采用模块化设计原则,
设计高内聚、低耦合的模块,
实现各自的功能。
场景是本文系统的核心资产,
各个模块基于场景实现功能实现和消息通讯。

上述设计目标及其对应的设计原则与工程体现关系如表\ref{tab:本文系统的设计目标和原则} 所示。

\begin{table}
	\centering
	\caption{本文系统的设计目标和原则}
	\label{tab:本文系统的设计目标和原则}
	\begin{tabular}{p{3cm}p{5cm}p{5cm}}
		\hline
		设计目标   & 设计原则         & 工程体现        \\
		\hline
		可持续性评估 & 数据驱动         & 场景资产数据库     \\
		可复现评估  & 解耦数据和计算、定式流程 & 设计独立的指标计算模块 \\
		可拓展性   & 模块化设计        & 模块之间不互相依赖   \\
		\hline
	\end{tabular}
\end{table}

\section{系统总体架构}
\label{sec:系统总体架构}

本节结合本系统的外部环境、
自身的功能和设计原则,
介绍系统总体架构。

\begin{figure}[ht]
	\centering
	\includegraphics[width=0.85\textwidth]{系统架构图.png}
	\caption{系统架构图}
	\label{fig:系统架构图}
\end{figure}

如图\ref{fig:系统架构图}所示,
本文系统外部面对两个系统,
集成化运行系统和可信展示平台,
前者承担着集成、配置、运行工具,
以及解析工具报告的功能;
后者承担着与测试人员互动、
展示工具指标的功能。
本文系统不直接对接测试人员,
而是面对这两个平台发出的各种消息命令。
面对可信展示平台,
本文系统接受各种场景管理命令,
提供工具指标数据;
面对集成化运行系统,
本文系统接受工具的检查报告并提供测试集。

为了满足系统的功能,
本文系统分为三个功能模块,
构建在一个数据库之上。
\textbf{场景工厂模块}负责生成场景,
为了保证场景生成质量,
聚焦在生成和检查任务上,
仅负责场景的增量动作
\textbf{场景资产管理模块}负责管理场景生命周期,
包括场景去重、分类等功能;
\textbf{工具指标计算模块}负责计算工具的两类指标,
前两个模块负责数据的治理层面,
计算模块则不对数据做任何修改,
仅基于数据进行指标计算和整理。
\textbf{场景知识库}负责承载场景空间和指标数据,
通过多个表的组织,
它不仅是测试用例的组织载体,
也是场景知识载体,
还是工具指标的检索载体;
三个功能模块并没有过程耦合,
都基于场景知识库的数据进行通信,
场景知识库在三个模块的配合下持续演进。

下面介绍本文系统各模块的设计,
其中章节\ref{sec:场景知识库}介绍场景知识库,
章节\ref{sec:场景工厂模块}介绍场景工厂模块,
章节\ref{sec:场景资产管理模块}介绍场景资产管理模块,

\section{场景知识库}
\label{sec:场景知识库}

% 回答为什么用知识库承载场景空间
根据章节\ref{sec:覆盖能力评估问题建模}对场景空间的定义解释和拓展方法,
承载场景空间的数据库要具备两方面的能力:
测试集组织和场景拓展参考,
前者是直接用于工具评估的数据,
后者是拓展场景时检索的基础。
要同时拥有这两个能力,
就需要设计完整的结构化数据。
如果仅以传统测试用例集合的形式存储场景,
将难以表达场景之间的语义关系以及规则、
缺陷与工具之间的关联信息,
不利于场景空间的持续拓展与能力评估结果的系统化分析。

% 使用数据作为通信桥梁
本文系统采用数据作为通信桥梁的设计原则,
在一个数据库中,
承载多种实体,
以场景空间为核心,
组织工具规则等对象,
以及各种实体之间的关系,
让实体之间的转换和功能模块相匹配。
经过分析整理,
本文将场景知识库的ER图设计如图\ref{fig:ER图}

\begin{figure}[ht]
	\centering
	\includegraphics[width=0.8\textwidth]{ER图.png}
	\caption{场景知识库ER图}
	\label{fig:ER图}
\end{figure}

对于场景知识库中关键的实体和关系的解释如表\ref{tab:场景知识库中关键的实体和关系}。

\begin{table}
	\centering
	\caption{场景知识库中关键的实体和关系}
	\label{tab:场景知识库中关键的实体和关系}
	\begin{tabular}{cc}
		\hline
		实体/关系 & 含义                            \\
		\hline
		场景    & 场景空间点的实例,用字段记录场景空间定位,是基本的测试单元 \\
		缺陷    & CWE\tc~相关缺陷                   \\
		规则    & 最小单位的测试对象                     \\
		工具    & 在规则的评估指标完成后将汇总计算工具的评估指标       \\
		覆盖    & 多对多的关系,工具对场景的实际告警结果           \\
		应当覆盖  & 多对多的关系,基于缺陷语义定义的规则期望覆盖关系      \\
		\hline
	\end{tabular}
\end{table}

% 自然语言作为桥梁的原因
场景需要被检索,
但是代码检索是另一个课题,
不是本文的重点。
本文有两类重要的知识需要互相检索,
一是代码文本,
二是场景的空间信息。
在正向梳理的时候,
需要从一个空间信息检索出相关的代码,
并且需要模糊检索,
以应对原有场景不足的情况;
在对场景进行空间定位的时候,
需要根据代码内容获取到空间信息。
在这两种情况下,
通过自然语言作为二者的桥梁非常合适。
自然语言的检索是非常成熟的,
已经有非常方便的检索工具,
代码和场景空间信息与自然语言的转换也比较直接,
因此本文选择使用自然语言作为桥梁,
图\ref{fig:自然语言作为检索桥梁}展示了其中的理由。

\begin{figure}[ht]
	\centering
	\includegraphics[width=0.8\textwidth]{自然语言作为检索桥梁.png}
	\caption{自然语言作为检索桥梁}
	\label{fig:自然语言作为检索桥梁}
\end{figure}

\section{场景工厂模块}
\label{sec:场景工厂模块}

% 场景工厂的职责、输入和输出
本文的场景工厂负责生成场景实例,
从架构图\ref{fig:系统架构图}可以看出,
输入是从可信平台传入的抽象约束,
其中包括目标场景的约束维度和目标缺陷,
输出是可评估的结构化场景候选,
模块同时必须承担质量控制责任,
生成结果要尽量符合静态分析工具检查的各种要求。
场景工厂只负责生成候选,
尽量保证场景质量。
总之,
场景工厂模块的生成对象是可被评估系统理解和消费的场景实例。

从设计角度出发,
为了保证场景工厂生成的质量和工程合法性,
本文在生成之外增加了一个校验环节,
这就是 RAG 加责任链的模式。

为了实现这样的设计,
本文设计了本模块的核心类图如图\ref{fig:场景工厂核心类图}。

\begin{figure}[ht]
	\centering
	\includegraphics[width=0.8\textwidth]{场景工厂核心类图.png}
	\caption{场景工厂核心类图}
	\label{fig:场景工厂核心类图}
\end{figure}

其中各个核心类的说明如表\ref{tab:场景工厂核心类说明}所示,

\begin{table}
	\centering
	\caption{场景工厂核心类说明}
	\label{tab:场景工厂核心类说明}
	\begin{tabular}{cp{9cm}}
		\hline
		类/接口           & 功能                           \\
		\hline
		RequestHandler & 整个评估系统的信息处理者,负责分发请求和向外界传递数据  \\
		Generator      & 场景工厂的主要功能类,负责生成流程的控制         \\
		Retriever      & RAG的核心检索类,可以根据信息高效检索最匹配的参考信息 \\
		Validator      & 责任链的抽象接口,每个责任节点都要实现其中的pass方法 \\
		Scenario       & 场景实例类,维护着一个实例实体              \\
		LLM            & LLM对话类,可以配置所需的大语言模型          \\
		\hline
	\end{tabular}
\end{table}

场景工厂内部仅维护生成上下文,
该上下文在生成完成后即被释放,
不对场景的后续生命周期负责,
校验是生成过程的内在质量约束。

从流程上说,
为了完成生成和校验的过程,
使用迭代生成和校验的方式提高质量,
图\ref{fig:场景生成活动图}展示的是场景工厂用来生成场景的核心活动图。

\begin{figure}[ht]
	\centering
	\includegraphics[width=0.85\textwidth]{场景生成活动图.png}
	\caption{场景生成活动图}
	\label{fig:场景生成活动图}
\end{figure}

对其中的一些环节作如下解释。
一次简单的LLM对话也可以生成一串代码,
但需要有检索和校验才能提升质量,
检索是通过提供已有的场景参考,
让LLM了解需要生成的场景的特点;
校验则使用一些辅助功能判断生成结果是否符合所需,
并提供错误反馈。
检索和校验是核心的质量保障。

一般的一次性RAG的检索和生成是连贯的,
但本文生成活动的检索内容是固定的,
而校验反馈不固定,
因此需要将检索和生成分隔开。
将生成过程放进一个循环中,
主要迭代的生成和校验。
迭代过程也需要设置一个次数上限,
用于防止无限迭代。

场景工厂不参与场景在空间中的定位和管理,
生成的内容有可能会让场景知识库的数据变得重复,
或者场景空间信息缺失。

\section{场景资产管理模块}
\label{sec:场景资产管理模块}

本文的场景资产管理模块负责管理场景生命周期,
为了解决场景工厂不能解决的问题,
本模块设计了场景添加、去重、分类、整理测试集等功能,
它是将场景空间整理为合格可使用的测试集的管家。

本文管理模块需要场景去重和分类。
重复的或同质的场景会影响评估指标的合理性,
而场景知识库并不会保证没有同质的场景,
所以需要根据场景的特征、空间位置等信息进行去重,
从而保证工具规则指标的准确。
分类功能是面向被动补充的,
在正向梳理的时候可以选择相应空间位置进行生成,
而面向被动缺陷,
需要将其放到场景空间的相应位置,
这就是用于空间定位的分类。

从架构图\ref{fig:系统架构图}可以看出,
本模块一方面输入可信平台传入的操作请求,
执行对场景知识库的直接动作;
另一方面提供集成化运行平台所需的测试集,
实际上就是场景知识库的全体有效场景集合,
因为计算工具的精准度等需要所有场景的检查结果。
接受返回的检查数据写进场景知识库,
为计算指标提供数据。
总之,
场景资产管理模块需要能提供直接可用的高质量测试集。

本模块的功能设计起来并不复杂,
但可以依赖本文系统的一些能力,
比如检索出分类参考等等。
本模块的核心类图如\ref{fig:场景资产管理核心类图}所示。

\begin{figure}[ht]
	\centering
	\includegraphics[width=\textwidth]{场景资产管理核心类图.png}
	\caption{场景资产管理核心类图}
	\label{fig:场景资产管理核心类图}
\end{figure}

其中各个核心类的说明如表\ref{tab:场景资产管理核心类说明}所示,
其中的功能并不复杂。

\begin{table}
	\centering
	\caption{场景资产管理核心类说明}
	\label{tab:场景资产管理核心类说明}
	\begin{tabular}{cc}
		\hline
		类/接口           & 功能                                   \\
		\hline
		RequestHandler & 整个评估系统的信息处理者,负责分发请求和向外界传递数据          \\
		Manager        & 场景资产管理的主要功能类                         \\
		Retriever      & RAG的核心检索类                            \\
		Classifier     & 分类器,输入的是已经入库的场景,需要使用Retriever的检索参考功能 \\
		Deduplicator   & 去重器,需要使用Retriever的检索参考功能             \\
		Scenario       & 场景实例类,维护着一个实例实体                      \\
		WeaknessSet    & 每个规则对应一个缺陷集,据此提供给工具测试集               \\
		LLM            & LLM对话类,可以配置所需的大语言模型                  \\
		\hline
	\end{tabular}
\end{table}

\section{工具指标计算模块}

本模块负责计算工具的两类指标,
一类是基于占比的宏观指标,
一类是基于能力边界的能力指标;
本模块不对数据做任何修改,
仅基于数据进行指标计算和整理。

从架构图\ref{fig:系统架构图}可以看出,
本模块并不输入集成化运行平台的检查结果,
而是从场景知识库中读取测试结果,
和集成化平台沟通的工作只交给场景资产管理模块。
本模块一方面基于测试结果计算得出指标,
然后写回场景知识库;
另一方面对于可信平台提供所需的指标数据。
总之,
本模块不对场景空间做任何的修改,
仅基于测试集的检查结果计算两类指标。

对于评估流程的调度机制,
为了实现自动化评估流程,
系统将本模块设计为自动化评估,
并不需要发出计算请求,
而是在场景资产管理模块完成计算结果写回后,
本模块自动计算指标;
另一方面,
本模块提供计算指标的查询接口,
供可信平台直接查询,
也不需要人工参与。

本模块的核心能力在章节\ref{sec:评估指标}和
章节\ref{sec:基于场景空间的测试结果解析}介绍了,
实现类图和流程图在章节\ref{sec:评估流程自动化实现}有详细介绍,
故此处不再赘述。

总结本章,
本文按照
需求分析、功能边界、设计原则、总体架构、模块设计
的思路,
完成了静态分析工具评估系统的概要设计,
其中各个模块都有着高内聚、低耦合的特征,
为下文实现该系统提供了设计基础。