% !TeX root = ../main.tex

\chapter{代码}

\begin{lstlisting}[caption={非平凡循环}, label={lst:非平凡循环}]
void func(int cond){
	int x = 0;
	while (cond){
		x++;
	}
	if (x > 5){
		bug(x);
	}
}
\end{lstlisting}

\begin{lstlisting}[caption={单调循环造成的路径增长}, label={lst:单调循环造成的路径增长}]
void func(){
	int i = 0;
	while (i < 1000){
		i++;
	}
	use(i);
}
\end{lstlisting}

\begin{lstlisting}[caption={语法可达但语义不可达}, label={lst:语法可达但语义不可达}]
void func(int x) {
    if (x > 10) {
        if (x < 5) {
            bug();
        }
    }
}
\end{lstlisting}


% todo!!!
\begin{lstlisting}[caption={场景1实际代码}, label={lst:场景1实际代码}]
void func(int cond){
	int x = 0;
	while (cond){
		x++;
	}
	if (x > 5){
		bug(x);
	}
}
\end{lstlisting}


\begin{lstlisting}[caption={场景生成提示词}, label={lst:场景生成提示词}]

--角色和执行--
你是一个对C/C++常见缺陷非常了解的专家,
请你根据描述生成一段代码片段。

--描述--
<description></description>

--要求--
1、要提供good code和bad code。
2、要能够通过编译,并准确表现描述中的缺陷内容。
3、不要输出多余解释。
4、只能有一个文件有main()函数。
5、在发生错误的地方注释错误原因。
6、根据具体场景判断文件是.c 还是.cpp
7、单个文件的输出格式如下:
<result>
{
"good_code":{
"file.c":"good code content",
}
"bad_code":{
"file.c":"bad code content",
}
}
</result>

8、多个文件的输出格式如下:
<result>
{
"good_code":{
"fileA.c":"good code content",
"fileB.c":"good code content
}
"bad_code":{
"fileA.c":"bad code content",
"fileB.c":"bad code content
}
}
</result>

--参考生成结果--
<reference></reference>

--上一次的结果,有不符合编译等错误,请参考改正--
<last_err></last_err>

\end{lstlisting}