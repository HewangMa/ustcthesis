% !TeX root = ../main.tex

\chapter{绪论}

% 多写就多写,以后再删

\section{研究背景与意义}

% 软件安全很重要,c/c++在软件中的占比大
随着软件在应用领域的广泛使用和迅猛发展,软件安全至关重要。上至国家安全,下至百姓生活,软件的缺陷都意味着或大或小的损失。在软件不断发展的过程中,减少软件缺陷是一个重大的课题。在软件语言家族中,C/C++占据比较大的比例,并且这类软件往往承担重要的低层功能,是构建操作系统、大型游戏引擎、数据库、嵌入式系统等对性能、效率和底层硬件控制有极致要求的软件的基石。C/C++软件的安全可信应该得到足够的重视。

% 静态分析是保障软件安全的重要一环
为了减少软件缺陷,静态分析技术被开发人员广泛使用。静态分析是指通过静态地检查代码本身而非运行程序,来分析软件的性质,静态分析可以应用在软件开发的各个阶段。静态分析技术依赖形式化的原理,可以实现快速分析,能够在分析精度和速度之间找到平衡。另外,由于软件缺陷发现的越晚,造成的成本越高,而静态分析技术能在软件开发多生命周期发挥作用,所以静态分析技术能够尽早地发现并帮助修复软件漏洞,以降低缺陷造成的开发成本和实质安全成本。由于速度快、不需要实际运行等优势,静态分析技术成为了C/C++软件的缺陷检测的常用工具,显著控制了软件缺陷。

% 提高静态分析质量和效率,能够帮助保障软件安全
在追求安全性和可靠性的大环境下,软件行业对清楚的了解静态分析工具的分析精度提出了更高的要求。高效充分评估静态分析工具的检测能力,是使用静态分析工具时的迫切需求,是保障软件质量的重要环节。只有深入了解工具的实际检测能力,才能在面对大规模代码的静态分析中,实现更加精准和高效的质量保障。

\subsection{C/C++静态分析工具及其应用现状}
% 有哪些工具? 应用现状如何?

学术界和工业界在C/C++静态分析领域已做了大量的研究,开发出了众多静态分析工具,在软件开发的过程中得到了广泛的应用。现有的静态分析工具众多,可以根据多个维度进行划分cite:

\begin{enumerate}
	\item \textbf{系统建模}。系统建模是静态分析工具对目标的抽象过程,即工具如何刻画目标程序,具体方法包括图结构、自动机结构、谓词抽象技术等等。
	\item \textbf{属性描述}。属性描述是静态分析工具如何建立不同的属性描述方法,即工具如何描述要检测的属性。具体来说,通常用有限自动机描述时序安全漏洞;用变量间的数值依赖关系和控制依赖关系精确刻画整数溢出、缓冲区溢出等。
	\item \textbf{检测过程}。检测过程指静态分析工具检测目标程序并得出分析结果的过程。常用的有基于词法模式匹配的方法、基于路径的方法、基于子图的方法、基于自动机的方法等。
	\item \textbf{结果验证}。结果验证指的是静态分析工具用何种方式去降低误报率、提高分析精度,通常有基于理论证明器的方法,基于SAT求解器的方法等。
\end{enumerate}

在不同的检查过程中,静态分析工具都有着不同的方法。众多工具遵循相同或不同的原理,针对软件缺陷检测提出他们自己的检查方法,比如下列几个工具在上述维度中的位置如表\ref{tab:position_of_some_tools}所示,其他工具总能在这四个维度上找到他们的位置。工具发展非常迅速,一次对现状的分类无法准确描述当前工具的多样。

\begin{table}
	\centering
	\bicaption
	{部分静态分析工具在四个维度中的位置}
	{Positions of Some Static Analysis Tools in Four Dimensions}
	\label{tab:position_of_some_tools}
	\begin{tabular}{ccccc}
		\hline
		工具         & 系统建模   & 属性描述   & 检测过程   & 结果验证   \\
		\hline
		Flawfinder & 字符串    & 字符串    & 模式匹配   & 无      \\
		UNO        & 自动机    & 自动机    & 自动机求积  & 无      \\
		Saturn     & 布尔程序   & 自动机    & 基于子图   & SAT    \\
		BLAST      & 谓词程序   & 自动机    & 基于路径   & SAT    \\
		\ldots     & \ldots & \ldots & \ldots & \ldots \\
		\hline
	\end{tabular}
\end{table}

尽管静态分析具备显著的优势,但其存在理论上的“不可能三角”:分析资源消耗、分析速度和分析精度三者不可同时达到最佳状态,只能在其中寻找平衡。这一限制决定了静态分析工具对代码问题的分析精度有不确定的表现,在面对企业级的庞大代码库时,表现出的能力往往是一个复杂的“黑盒”。对C/C++静态分析工具来说,特别是面对多文件交叉、函数调用层次较深或控制流程复杂的代码时,各种工具的检测能力差异尤为显著。

\subsection{静态分析工具评估的国内外研究现状}

% 评估是本论文的关注点,那么别人对评估的研究现状如何?

目前研究人员对静态分析技术的研究比较充足,但是对静态分析工具的评估研究相对较少。首先是对静态分析工具的评估局限于一些静态且单次的测试,这不能随着工具的更新和发展准确的表现工具能力。

\subsubsection{一次性评估结果}
有较多研究集中在对工具进行一次性评估。文献。。。。文献。。。

\subsubsection{评估框架}
目前评估框架有如下研究。。

\subsubsection{缺陷测试集}
测试集是用于评估静态分析工具的最直接的方法。包括。。。

\subsection{研究意义}
对静态分析工具进行评估,研究评估方法,设计出一套模块化评估系统据


\section{本文主要工作内容}
\section{本文工作的创新性及主要技术特色}
\section{论文组织与章节结构安排}

本文共分为六个章节。

第一章为绪论。本章首先介绍了针对C/C++语言的静态分析在软件行业中的应用和研究现状,简单阐述了对静态分析工具进行系统化评估的重要性,然后介绍了本文的主要工作和组织架构。

第二章为研究现状。本章对静态分析和代码生成的研究现状进行罗列和分析,分析他们的不足之处和可以进行结合的点。本章为本文的评估方法和系统设计作理论铺垫。

第三章为场景化覆盖能力评估问题建模。本章提出一种场景化的评估方法,研究了静态分析工具的原理和对应的能力边界,基于能力边界引出场景空间的结构化建模方法,通过形式化的建模过程阐述了场景化评估方法的合理性和系统性,为静态分析工具的评估系统设计提供理论基础,并分析了这种基于场景的评估方法的优势与局限。

第四章为面向场景的评估系统。本章基于前文对评估问题的建模,设计出一套模块化评估系统,基于一个场景数据库,设计两个模块分别进行生成和管理,并介绍了评估指标计算方法,最后介绍了评估系统在实现中的具体技术手段和模块化优势。

第五章为实验和分析。本章针对静态分析评估系统的目标需求,设计了针对性的实验,使用clang-tidy作为实验静态分析工具,将其规则进行细致评估和分析,呈现系统的效果和多工具可拓展性。最后基于实验结果C/C++静态分析工具评估问题进行了优势和不足分析。
