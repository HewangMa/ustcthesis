% !TeX root = ../main.tex
\chapter{绪论}

\section{研究背景与意义}
% 软件安全很重要,\tc~在软件中的占比大
软件在应用领域使用广泛、发展迅猛,
软件的安全可信一直被研究人员高度重视。
缺陷(Weakness)指软件或程序中存在的某种破坏正常运行能力的问题、错误。
上至国家安全,下至百姓生活,
软件的缺陷都意味着或大或小的损失。
在软件不断发展的过程中,
减少软件缺陷是一个重大的课题。
在软件语言家族中,\tc~占据比较大的比例,
并且这类软件往往承担重要的低层功能,
是构建操作系统、大型游戏引擎、数据库、
嵌入式系统等对性能、效率和底层硬件控制有极致要求的软件的基石。
\tc~软件的安全可信至关重要。

% 静态分析是保障软件安全的重要一环
为了减少软件缺陷,静态分析技术被开发人员广泛使用。
静态分析是指通过静态地检查代码本身而非运行程序,
来分析软件的性质的过程,静态分析技术可以应用在软件开发的各个阶段。
静态分析技术依赖形式化的原理,可以实现快速分析,
能够在分析精度和速度之间找到平衡。
另外,由于软件缺陷发现的越早,造成的成本越低,
而静态分析技术能在软件开发多生命周期发挥作用,
所以静态分析技术能够尽早地发现并帮助修复软件缺陷,
以降低缺陷造成的开发成本和实质安全成本。
由于速度快、不需要实际运行等优势,
静态分析技术成为了\tc~软件的缺陷检测的常用工具,
显著控制了软件缺陷。

% 要提高静态分析质量和效率
在追求安全性和可靠性的大环境下,
软件行业对清楚的了解静态分析工具的分析精度提出了更高的要求。
每个软件缺陷缺陷都对应着众多场景,
\textbf{场景}(Scene)描述了缺陷发生的实际条件和路径。
根据静态分析的莱斯定理(Rice's Theorem)\cite{riceTheory},
静态分析工具永远不能保证检测出软件的所有缺陷场景。
但为了尽可能减少软件缺陷,尽量覆盖更多场景,
静态分析工具能力提升一直为研究人员所重视。

\subsection{静态分析工具评估的国内外研究现状}
\label{sec:静态分析工具评估的国内外研究现状}

% 评估是本论文的关注点,那么别人对评估的研究现状如何?
静态分析工具的能力是是需要不断进步的,对工具能力的评估就十分重要。
只有了解静态分析工具,清楚工具能力,软件从业者才能更好的利用工具保障软件安全,
而仅凭静态分析工具公布的说明文档、或学术论文对工具源头技术的探讨难以清楚地呈现工具能力。
相对于对静态分析技术的研究,目前研究人员对静态分析工具的评估研究相对较少。

% 一次性对工具现状的评估
目前国内外对静态分析工具的评估研究有一些一次性对工具现状的评估,
以期直接提供静态分析工具的选择建议。
文献\cite{XNCJ201012002044}将\tc~的软件缺陷分类为8类,
对Coverity Prevent、ITS4 Klocwork K7、SPLINT
等常见的\tc~静态分析工具进行了定性的评估。
作者指出Coverity Prevent分析能力强,适合控制流信息分析,
但变量属性之间的关系复杂时,不容易判断。
作者总结道,为了减少软件缺陷,
可以综合多个静态分析工具进行检查,或结合动态分析和静态分析。

文献\cite{QBZH201102032}对几种主流 C 语言静态分析工具的功能性进行了对比研究,
作者着重评估了Coverity Prevent、Klocwork K7、Spiint等
工具在重点的C语言缺陷如空指针引用、悬空指针、资源泄露
等缺陷上的功能性,
指出这些工具都在检查软件缺陷上有各自的优势,
但都不能完全成为软件安全的依赖。

文献\cite{6987569}基于常见缺陷分类和
源代码弱点定义了常见的软件弱点,
对 Yasca、CAT.NET 和 FindBugs 等非商业静态分析工具进行了一次性评估。
作者指出安全静态分析工具在一定程度上能够有效检测源代码中的安全漏洞、
源代码分析器比字节码和二进制代码扫描器能够检测到更多弱点。
但虽然这些工具可以辅助开发团队进行安全代码审查,
它们也不足以发现软件中所有常见的弱点。

文献\cite{JZCK202412040}整理了基本的静态分析工具评估方法,
在linux系统环境下对
CppCheck、Flawfinder 和 TscanCode 三个静态分析工具进行分析。
作者指出CppCheck工具在综合评估中表现最好,
整体的缺陷分析性能在3个静态分析工具中最佳,
而Flawfinder和TscanCode类似,
都采用了专注于改进误报或漏报,同时选择放弃另一个,
但是都同样没有取得较好的效果。

文献\cite{1018997162.nh}整理了测试集和一些评估方法,
对Cppcheck、Clan、Frama-C和一个商业工具进行了一次评估。
作者使用多种评估维度,对这四个工具进行了多个维度的定量比较,
发现这四个工具在不同维度上的能力不同,
并且它们覆盖缺陷类别也不同,
因为在实际使用的时候不能依赖一个工具尝试保证整个软件的安全。
作者同时研究了工具集成和测试集拓展,
提出基于人工审查缺陷报告并提取核心代码的方法,
这是一种人工驱动的单点式测试集拓展,
这都对静态分析工具评估问题有所启发。

% 评估框架
除了对静态分析工具的一次性评估,也有一些研究对评估框架和评估方法进行了探索。
文献\cite{1021682624.nh}发现很少有研究去检验商业和开源扫描仪的有效性。
因此,作者提出了一个自动评估开源Web扫描程序漏洞严重性的框架,
提出了一种帮助安全专家选取最适合Java代码文件的缺陷检测工具方法,
对八种广泛使用的网络应用漏洞检测工具进行了实证评估,
并在评估指标上引入了缺陷场景危害严重程度、
工具检测性能等更多维度,并在此基础上设计了一套自动评估框架,
以期增强商业和开源web漏洞扫描器的有效性。

文献\cite{XWOS4F8A48F6DAF179A6F8DB0DEF80877E39}
指出以往的工具设计中的具体缺陷或不足之处往往不为人知或缺乏充分的文档记录,
导致研究人员、开发者和用户对其抱有不切实际的信任。
设计了一种基于突变的评估框架,名为μSE,该框架利用成熟的变异分析方法,
结合基于编译和执行标准的过滤技术,减少无效突变,
系统地评估Android静态分析工具,
以发现、记录和修复缺陷,
并将其应用于一系列用于检测应用中隐私数据泄露的知名Android静态分析工具。

% 测试集拓展
另一方面,也有一些研究对测试集拓展进行了探索和研究,
文献\cite{1018997162.nh}人工审查了部分工具在部分开源和闭源代码上的缺陷报告,
记录了其中的缺陷原因、缺陷核心代码等,
并对三个工具进行了评估。
作者实现了专用的缺陷爬取程序,并提供了可拓展接口,
方便后人实现更快速的测试集持续扩展。

文献\cite{10.1145/3524610.3527899}研究了静态分析工具包含的现有规则本身存在的错误,
提出了一种差异化测试方法,
通过映射两个不同静态代码分析工具中相似的缺陷规则,
并使用相同的测试用例来比较两种工具对这些规则的输出结果,
从而发现潜在的误报和漏报。
作者检测了四种广泛使用的静态缺陷查找工具规则中的缺陷,
并对发现的缺陷进行了定性研究。

文献\cite{1024508297.nh}总结分析了主流静态代码分析工具的缺陷规则与缺陷案例,
细化归纳出七个主要的代码主要特性类别:
语言特征与结构、
表达式与运算、
变量与数据、
控制流与异常处理、
面向对象的特性运用、
库与 API 的应用
以及代码的组织和设计模式,
基于此构建蜕变关系,
设计实现四条测试用例变换规则,
包括为方法添加嵌套类、
匿名类转 Lambda 表达式、
基于表达式的等价变换等操作,
为静态分析工具评估的测试集拓展提供了有效方法和路线。
作者同时对相关的静态分析工具进行了评估和研究。

总体而言,当前国内外研究多集中于静态分析工具的一次性评估,
偏重主流工具的功能和检测能力对比,
方法主要依赖现有测试集或手工构建特化用例,
旨在使用实际检测的方式描述当前各种主流工具的检测能力。
虽然此类研究能为工具选型提供参考,但缺乏持续性和广泛性,
难以适应工具和软件产品的快速迭代。
另一方面,有研究对静态分析工具的评估测试集及其拓展进行了尝试和研究,
贴合当前工具的能力现状,有一定的持续性,
但拓展方法往往基于当前静态分析工具的能力针对性拓展,
在测试集动态更新及评估结果长期管理方面存在不足,
尚难满足高安全性软件行业对静态分析工具全面、动态能力验证的实际需求。

\subsection{研究意义}

% explain 更加系统化的静态分析工具评估的重要性
在提升静态分析工具分析精度的道路上,
准确清晰地评估静态分析工具的能力现状也相应的变得非常重要。
然而目前的评估工作还不足以让研究人员深入了解所有工具的实际检测能力,
并且在工具和软件缺陷更新的时候持续跟踪工具能力。

系统化的静态分析工具能力评估,
可以适用于所有的静态分析工具,
并且可以持续的评估下去,
而不应该针对一个工具、进行一次性评估。

系统化的为测试人员选择一个或一套精确的静态分析工具,
可以让他们直观的看到静态分析工具的能力。
这样不仅能够指导软件缺陷检查工作,
也对静态分析工具的研究人员提供准确的优化目标,
更能向外界提供足够的材料支撑,
说明自身软件的安全性。

基于前文对静态分析工具和评估的背景介绍,
静态分析工具的评估对软件安全有长远而重要的意义。
本文旨在基于前人对静态分析工具评估研究的基础上,
将该问题进一步系统化,并结合当前的前沿技术加以实践,
设计并实现一套\tc~静态分析工具自动化评估系统。

具体来说,本文的研究意义主要在以下几方面。

% todo
\iffalse
	\begin{enumerate}
		\item 用系统化的思路对静态分析工具的评估测试用例拓展问题进行建模,
		      对场景这个概念进行更系统和清晰的定义,
		      将缺陷和场景的关系进一步讨论,
		      设计出适合静态分析工具的
		      让测试集的拓展可以更准确的覆盖到更广的缺陷场景
		\item sd
	\end{enumerate}

	要设计一套新的评估框架和测试集拓展框架,
	针对
	并设计和实现一种场景驱动的对\tc~静态分析工具分析精度的自动化评估系统,
	基于场景动态地评估静态分析工具的分析精度,
	帮助企业软件开发中静态分析技术的高效应用。
	本课题的系统对于企业软件开发的质量保障和行业对静态分析工具的应用具有深远意义。

	其一,该系统通过场景驱动的详细检验,
	触及静态分析工具在不同场景的能力边界,
	能够在企业对自身使用的静态分析工具具备场景级的认知能力,
	从而更好的选择和优化静态分析工具。

	其二,该系统的评估结果从正反两面提升软件可信度:
	一方面,所有能够被静态分析工具检测到的问题都已被发现并修复;
	另一方面,对于工具无法覆盖的场景,
	通过对已有场景的分析和验证,
	确保相关问题在代码中并不存在。
	有了本系统对静态分析工具的检查能力说明,
	企业可以更好的展示说明以上正反面。

	其三,该系统不是一次性的对静态分析工具的打分,
	而是让测试集紧跟企业产品进行拓展,
	让工具更新直接反馈其能力,
	从而摈除对工具能力认知的滞后性,
	让企业在完成产品时及时完成产品的检查和安全说明。

	总而言之,设计和实现场景驱动的\tc~静态分析工具自动化评估系统,
	为企业向外界证明其产品安全可信性提供了有力依据,
	补足了行业内对于持续评估静态分析工具能力的不足。
\fi

\section{本文主要工作内容}

todo

\section{本文工作的创新性及主要技术特色}

todo

\section{论文组织与章节结构安排}

本文共分为个章节。

第一章为绪论。
在软件安全尤为重要的背景下,
首先介绍了针对\tc~语言的静态分析工具在软件行业中的应用现状,
然后介绍了静态分析工具评估的国内外研究现状,
简单阐述了对静态分析工具进行系统化评估的重要性。
然后介绍了本文的研究意义、主要工作、创新点和技术特色,
最后介绍了本文的主要工作和组织架构。

第二章为本文相关技术。
本章对本文研究所用到的前沿技术背景进行介绍,
包括静态分析评估的相关方法和指标、
代码生成和容器化的前沿技术,
为后文的设计和实现提供技术路线支撑。

第三章为覆盖能力评估问题建模,
本章通过对静态分析工具的能力进行多维度分析,
介绍了场景空间,
以及场景的扩展方式,
然后介绍了宏观指标和工具能力边界评估方法,
最后介绍了基于场景空间的评估方法的优势和不足。

% todo
\iffalse
	第四章为面向场景的评估系统。
	本章基于前文对评估问题的建模,设计出一套模块化评估系统,
	基于一个场景数据库,设计两个模块分别进行生成和管理,
	并介绍了评估指标计算方法,最后介绍了评估系统在实现中的具体技术手段和模块化优势。

	第五章为实验和分析。
	本章针对静态分析评估系统的目标需求,设计了针对性的实验,
	使用clang-tidy作为实验静态分析工具,将其规则进行细致评估和分析,
	呈现系统的效果和多工具可拓展性。
	最后基于实验结果\tc~静态分析工具评估问题进行了优势和不足分析。
\fi

