% !TeX root = ../main.tex

\chapter{绪论}

% 多写就多写,以后再删

\section{研究背景与意义}

% 软件安全很重要,\tc~在软件中的占比大
随着软件在应用领域的广泛使用和迅猛发展,软件安全至关重要。
缺陷(Weakness)指软件或程序中存在的某种破坏正常运行能力的问题、错误。
上至国家安全,下至百姓生活,软件的缺陷都意味着或大或小的损失。
在软件不断发展的过程中,减少软件缺陷是一个重大的课题。
在软件语言家族中,\tc~占据比较大的比例,并且这类软件往往承担重要的低层功能,
是构建操作系统、大型游戏引擎、数据库、嵌入式系统等对性能、效率和底层硬件控制有极致要求的软件的基石。
\tc~软件的安全可信应该得到足够的重视。

% 静态分析是保障软件安全的重要一环
为了减少软件缺陷,静态分析技术被开发人员广泛使用。
静态分析是指通过静态地检查代码本身而非运行程序,来分析软件的性质的过程,静态分析技术可以应用在软件开发的各个阶段。
静态分析技术依赖形式化的原理,可以实现快速分析,能够在分析精度和速度之间找到平衡。
另外,由于软件缺陷发现的越早,造成的成本越低,而静态分析技术能在软件开发多生命周期发挥作用,
所以静态分析技术能够尽早地发现并帮助修复软件缺陷,以降低缺陷造成的开发成本和实质安全成本。
由于速度快、不需要实际运行等优势,静态分析技术成为了\tc~软件的缺陷检测的常用工具,显著控制了软件缺陷。

% 提高静态分析质量和效率,能够帮助保障软件安全
由于静态分析工具的广泛使用和缺陷检测的依赖性,在追求安全性和可靠性的大环境下,
软件行业对清楚的了解静态分析工具的分析精度提出了更高的要求。
静态分析工具并没有保证检测出所有缺陷的能力,对于软件缺陷的保障来说,高效充分评估静态分析工具的检测能力,
是使用静态分析工具时的迫切需求,是保障软件质量的重要环节。
测试人员需要深入了解工具的实际检测能力,了解工具对具体程序问题的覆盖能力,
才能在面对大规模代码的静态分析中,实现更加精准和高效的质量保障。




\subsection{\tc~静态分析工具及其应用现状}

% 工具分类
学术界和工业界在\tc~静态分析领域已做了大量的研究,开发出了众多静态分析工具,并且发展非常迅速。
如今商业工具有LDRA Testbed、Perforce Klocwork、Parasoft \tc~test、Synopsys Coverity等;
开源工具有Flawfinder、FindBugs、SonarQube、Clang-tidy等。许多公司也在研究开发属于自己的静态分析工具。
对于如此众多的静态分析工具,可以根据多个维度进行划分:

\begin{enumerate}
	\item \textbf{系统建模}。系统建模是静态分析工具对目标的抽象过程,即工具如何刻画目标程序,
	      具体方法包括图结构、自动机结构、谓词抽象技术等等。
	\item \textbf{属性描述}。属性描述是静态分析工具如何建立不同的属性描述方法,即工具如何描述要检测的属性。
	      具体来说,通常用有限自动机描述时序安全缺陷;
	      用变量间的数值依赖关系和控制依赖关系精确刻画整数溢出、缓冲区溢出等。
	\item \textbf{检测过程}。检测过程指静态分析工具检测目标程序并得出分析结果的过程。
	      常用的有基于词法模式匹配的方法、基于路径的方法、基于子图的方法、基于自动机的方法等。
	\item \textbf{结果验证}。结果验证指的是静态分析工具用何种方式去降低误报率、提高分析精度,
	      通常有基于理论证明器的方法,基于SAT(Satisfiability)求解器的方法等。
\end{enumerate}

在不同的检查过程中,静态分析工具都有着不同的方法。
众多工具遵循相同或不同的原理,针对软件缺陷检测提出他们自己的检查方法,
比如下列几个工具在上述维度中的位置如表\ref{tab:position_of_some_tools}所示,
其他工具总能在这四个维度上找到他们的位置。
工具发展非常迅速,一次对现状的分类无法准确描述当前工具的多样。

\begin{table}
	\centering
	\caption
	{部分静态分析工具在四个维度中的位置}
	\label{tab:position_of_some_tools}
	\begin{tabular}{ccccc}
		\hline
		工具         & 系统建模   & 属性描述   & 检测过程   & 结果验证   \\
		\hline
		Flawfinder & 字符串    & 字符串    & 模式匹配   & 无      \\
		UNO        & 自动机    & 自动机    & 自动机求积  & 无      \\
		Saturn     & 布尔程序   & 自动机    & 基于子图   & SAT求解器 \\
		BLAST      & 谓词程序   & 自动机    & 基于路径   & SAT求解器 \\
		\ldots     & \ldots & \ldots & \ldots & \ldots \\
		\hline
	\end{tabular}
\end{table}


% 广义的SAT的不足
尽管静态分析具备显著的优势,但也有很多广义上的不足。
首先,静态分析存在理论上的“不可能三角”:分析资源消耗、分析速度和分析精度三者不可同时达到最佳状态,
只能在其中寻找平衡。这一限制决定了静态分析工具对代码问题的分析精度有不确定的表现,
在面对企业级的庞大代码库时,表现出的能力往往是一个复杂的“黑盒”。
其次,静态分析技术在实际应用中也会产生诸如误报率高、精确度较低、告警合并和处理比较复杂等问题。
最后,为了更好的保障安全,大规模软件在进行静态分析检测的时候,往往会结合多个工具,
在工具本身的检测精度之外,还有告警处理、数据整合等问题会影响实际应用效果。

% c/cpp对SAT的分析带来的不足
对\tc~静态分析工具来说,也存在基于语言特性带来的不足。
第一,\tc~语言因其底层操作能力强、资源控制精细、广泛用于企业级大规模软件开发,
因此一次完整的静态分析可能长达若干小时。
第二,由于各种工具的检察原理不同,面对多文件交叉、函数调用层次较深或控制流程复杂的代码时,
各种静态分析工具的检测能力差异尤为显著。
第三,\tc~静态分析工具需要面对内存越界访问、资源泄漏、空指针访问、\tc~代码规范、以及定制化检查规则,
这些特殊的缺陷往往有特化的触条件。
具体来说,各种工具在面对如下\tc~问题会有明显的检查瓶颈:

\begin{enumerate}
	\item \textbf{跨文件}。跨文件的程序会大幅增加代码的规模,
	      涉及到的全局变量、函数和类的数量众多,这使得分析复杂度呈指数级增长。
	      跨文件调用的上下文可能丢失,如函数的定义和调用在不同文件中,
	      参数的精确值和调用栈的信息很难还原。

	\item \textbf{函数调用层数深}。调用深层次的函数时,涉及多层函数间的调用关系,增加了解析每层语义的难度。
	      另外,静态分析工具对函数调用通常采用上下文敏感分析,即根据具体的调用点来分析函数行为。
	      如果调用层次过深,则需要追踪调用链中的所有上下文信息,分析过程会更加复杂且容易失去精度。
	      另外在深层调用中,递归分析的复杂度急剧增加,可能无法准确推导出函数最终的影响。

	\item \textbf{程序基本块太多}。程序基本块是一段线性的程序码,只能从这段程式码开始处进入这段程序。
	      基本块太多会导致路径组合爆炸,在静态分析中,基本块多意味着可能的路径数急剧增加。
	      如果程序中有许多分支和循环,每条路径都需要单独分析,导致路径数量指数增长,
	      进而导致时间和空间开销过大,在不可能三角的平衡下丢失精度。

	\item \textbf{指针断链}。指针断链指的指针原本指向一个有效的内存地址,
	      但后来所指的内存被释放或失效后,指针依然保持原来的值并试图被访问。
	      分析工具需要在跨函数、跨模块时保持准确的指针追踪,
	      单独分析某一个部分通常无法捕获完整的指针使用语境,
	      导致程序分析工具在面对指针断链问题时展现出不足。

	\item \textbf{代码重入次数(max loop num,MLN)太大}。
	      大多数静态分析工具在对循环进行检查时,往往使用展开方式,
	      将循环展开 MLN 次,以检查循环代码在重入多次时的安全问题,
	      但该参数对于检查性能来说影响较大,静态分析工具一般将它设置为较小值,
	      对于确保重入代码的安全性来说保障性有待检验。
\end{enumerate}





\subsection{静态分析工具评估的国内外研究现状}

% 评估是本论文的关注点,那么别人对评估的研究现状如何?
静态分析工具的能力是是需要不断进步的,对工具能力的评估就显得十分重要。
只有了解静态分析工具,清楚工具能力,软件从业者才能更好的利用工具保障软件安全。
但仅凭静态分析工具公布的说明文档、或学术论文对工具源头技术的探讨难以清楚地呈现工具能力。
目前研究人员对静态分析技术的研究比较充足,但是对工具的评估研究相对较少。
一般来说,对静态分析工具的评估大都使用测试集实测的方式,评估方法如下。

\begin{enumerate}
	\item \textbf{明确评估对象与测试集}。
	      测试集指的是对工具进行评估时准备的缺陷或测试用例全集。
	      定义被评估的静态分析工具和测试集,设定衡量工具能力的评估指标。
	      例如,测试哪些具体测例文件,并通过哪些标准衡量性能。
	      格式化存储基准测试集中的关键信息,例如文件数量、缺陷类型、注入缺陷位置等,
	      为后续评估过程提供数据支持。

	\item \textbf{运行静态分析工具并获取原始结果}。
	      利用自动化工具高效批量执行测试,采集分析结果,
	      提高运行效率的同时减少手动操作带来的误差。

	\item \textbf{解析结果并计算评估指标}。
	      对原始结果中的告警信息进行处理与分类,计算并对比各类评估指标,
	      对工具能力作出全面分析。同时在这一步需要对缺陷具体代码进行归档。
\end{enumerate}





\subsubsection{测试集}

测试集是用于评估静态分析工具的基础内容。
\textbf{场景}(Scene)描述了缺陷发生的实际条件和路径,一种缺陷可以关联多个场景。
测试集的完整性和复杂性在静态分析工具能力评估中尤为关键,它需要能够体现实际软件的复杂场景。
若测试集包含的缺陷场景有限,评估结果可能无法反映工具在真实环境中的表现,导致实际应用中的缺陷频发。


\textbf{常见缺陷枚举}(Common Weakness Enumeration,CWE)
是开源社区维护的常见软件与硬件弱点列表,它并非仅针对静态分析工具,而是涵盖广义的软件缺陷集合。
CWE 共包含 944 种计算机软硬件相关缺陷,其中软件领域缺陷 399 种,与\tc~相关的缺陷共 118 种。
对于所有缺陷,CWE 从分类层级、基础类型、变量、作用链等多个维度进行组织;
但针对\tc~相关的 CWE 缺陷,本文将这 118 种缺陷视为并列的缺陷集合进行分析。
每个 CWE 缺陷均具有唯一编号,常见的\tc~缺陷包括
CWE-126:\texttt{Buffer\_OverRead}(越界读),
CWE-416:\texttt{Use\_After\_Free}(释放后使用)等。


\textbf{朱丽叶测试集}(Juliet Test Suite,JTS)
是由美国国家安全局(NSA)软件质量中心(CAS)为评估静态分析工具的能力而创建的专用测试集,
它基于CWE缺陷分类,由 61387 个测试用例组成,大部分测试用例使用定制测试用例模板引擎生成。
其他部分的测试用例由人工创建而成。
该测试集涵盖了 118 种CWE类型,由 96896 个\tc~文件、867 万行代码构成。

\textbf{Functional类型}是JTS在同种CWE缺陷的基础上对缺陷类型的进一步细分。
每个CWE缺陷的Functional类型不同,如果该 CWE 类型没有更细粒度的内容,则 Functional 类型为 basic。
测试用例中使用的数据结构、特殊函数或特殊标记都有可能是Functional类型中的关键词。
举例来说,
CWE-126: \texttt{Buffer\_OverRead} 越界读内存这个缺陷,
就包括 \texttt{char\_alloca\_loop}、\texttt{char\_alloca\_memcpy}、\texttt{char\_alloca\_memmove}
等多种functional类型。

\textbf{代码结构复杂度类型}(Flow)是JTS设计的代码结构中的数据流类型、控制流类型。
这是另一种对缺陷的分类维度,更针对静态分析的工具原理,能够对工具的能力进行更细致的划分。
Flow类型共有48种,用数字标记,01代表baseline,02至30代表控制流,31至84代表数据流。
其中部分列举如下。

\begin{table}
	\centering
	\caption
	{部分Flow类型含义}
	\label{tab:Meaning_of_some_Flow_types}
	\begin{tabular}{ccp{9cm}}
		\hline
		Flow 类型 & 所属类别    & 该 Flow 类型代码的主要结构特点                                                                                                           \\
		\hline
		02      & 控制流     & \texttt{if(1) and if(0) }                                                                                                    \\
		08      & 控制流     & \texttt{if(staticReturnsTrue()) and if(staticReturnsFalse()) }                                                               \\
		14      & 控制流     & \texttt{if(globalFive==5) and if(globalFive!=5) }                                                                            \\
		18      & 控制流     & \texttt{goto statements}                                                                                                     \\
		31      & 数据流     & \texttt{Data flow using a copy of data within the same function}                                                             \\
		33      & 数据流     & \texttt{Use of a C++ reference to data within the same function }                                                            \\
		65      & 控制流、数据流 & \texttt{Data passed as an argument from one function to a function in a different source file called via a function pointer} \\
		\ldots  & \ldots  & \ldots                                                                                                                       \\
		\hline
	\end{tabular}
\end{table}





\subsubsection{评估指标}

静态分析工具的性能分为很多维度,用对应的指标来表征。
在执行完检查之后,指标通过“有无告警”和“有无缺陷”矩阵数据
表\ref{tab:statistics_of_analysis}来计算。

\begin{table}
	\centering
	\caption
	{静态分析检查得到的数据}
	\label{tab:statistics_of_analysis}
	\begin{tabular}{p{2cm}p{5cm}p{5cm}}
		\hline
		    & 有告警                   & 无告警                   \\
		\hline
		有缺陷 & TP(True Positive,正报)  & FN(False Nagetive,漏报) \\
		无缺陷 & FP(False Positive,误报) & TN(True Nagetive)     \\
		\hline
	\end{tabular}
\end{table}

\textbf{精确度(Precision)}指工具所有告警中实际为缺陷的比例,
其计算方法如(\ref{eq:precision})所示。
精确度回答了“静态分析工具的缺陷告警是否精确”的问题,描述了静态工具的可信程度。
精确度越高,代表静态分析工具给出的告警更有可能确实是缺陷(即该告警的可信度越高)。

\begin{equation}
	\label{eq:precision}
	\text{Precision} = \frac{TP}{TP + FP}
\end{equation}

\textbf{召回率(Recall)}指工具正确告警缺陷占所有缺陷的比例,
其计算方法如(\ref{eq:recall})所示。
召回率回答了“静态分析工具能识别多少缺陷”的问题,
描述了工具告警对于整体缺陷的覆盖程度,反映工具的漏报情况。
召回率越高,代表静态分析工具找到了较多的缺陷,漏报率低。

\begin{equation}
	\label{eq:recall}
	\text{Recall} = \frac{TP}{TP + FN}
\end{equation}

\textbf{F1分数(F1)}是精确度和召回率的调和平均,为综合指标,
其计算方法如(\ref{eq:f1})所示。
F1 分数反映静态分析工具在漏报和误报之间的平衡性。
一个工具如果采用激进策略,选择将所有疑似缺陷都进行告警,
则该工具尽管拥有非常高的召回率,但其精准度会非常低,
在这种情况下,F1 分数可以更好地反映工具的缺陷分析能力。

\begin{equation}
	\label{eq:f1}
	\text{F1} = \frac{2 * Precision * Recall}{Precision + Recall}
\end{equation}

\textbf{区分度(Discrimination Rate)}
被用来评估静态分析工具辨别“有缺陷程序结构”和“无缺陷程序结构”的能力,
其计算方法如(\ref{eq:discrimination})所示。
NSA CAS(美国国家安全局 软件质量中心)引入了“有区分度的正报”这个概念,
即Discriminations,其含义是工具正确地报出某测试用例的“有缺陷函数”,
同时没有错误地报出该测试用例的“无缺陷函数”。
对于这种告警,开发人员再修改正确之后告警会自动消失。

\begin{equation}
	\label{eq:discrimination}
	\text{Discrimination} = \frac{Discriminations}{TP + FN}
\end{equation}

\textbf{CWE覆盖度(CWE Coverage)}是工具能够检测的缺陷类数和总缺陷类数的比值,
用来评估静态分析工具在缺陷粒度下的覆盖能力。

为此行业中有一个指标为CWE覆盖度(Coverage for CWE),
它是静态分析工具可以检出的缺陷类型数和测试集缺陷类型总数的比[5],
CWE 覆盖度 = 工具正确缺陷告警的CWE类型个数 / 测试集缺陷类型总数

% \begin{equation}
% 	\label{eq:discrimination}
% 	\text{Discrimination} = \frac{Discriminations}{TP + FN}
% \end{equation}

这些指标表征了静态分析工具正确检测软件缺陷的能力,
是工具的核心评估指标。
其他评估指标还有检查速度、运行时算力消耗、检查功能等等,
也是静态分析工具在使用时的指标,但不属于本文研究的对象。

\subsubsection{一次性评估结果}

目前国内外对静态分析工具核心指标的评估研究总量不多,
主要集中于一次性对工具现状的评估。


首先是对静态分析工具的评估局限于一些静态且单次的测试,
这不能随着工具的更新和发展准确的表现工具能力。
其次是大部分静态分析评估都在仅限于少量工具,不能代表目前软件行业使用的静态分析工具全体,
对于测试人员来说,要参考这些评估结果只是大概了解几个工具的几个能力而已。

\subsection{研究意义}
对静态分析工具进行评估,研究评估方法,设计出一套模块化评估系统据


\section{本文主要工作内容}
\section{本文工作的创新性及主要技术特色}
\section{论文组织与章节结构安排}

本文共分为六个章节。

第一章为绪论。
本章首先介绍了针对\tc~语言的静态分析在软件行业中的应用和研究现状,
简单阐述了对静态分析工具进行系统化评估的重要性,
然后介绍了本文的主要工作和组织架构。

第二章为研究现状。
本章对静态分析和代码生成的研究现状进行罗列和分析,
分析他们的不足之处和可以进行结合的点。
本章为本文的评估方法和系统设计作理论铺垫。

第三章为场景化覆盖能力评估问题建模。
本章提出一种场景化的评估方法,研究了静态分析工具的原理和对应的能力边界,
基于能力边界引出场景空间的结构化建模方法,
通过形式化的建模过程阐述了场景化评估方法的合理性和系统性,
为静态分析工具的评估系统设计提供理论基础,
并分析了这种基于场景的评估方法的优势与局限。

第四章为面向场景的评估系统。
本章基于前文对评估问题的建模,设计出一套模块化评估系统,
基于一个场景数据库,设计两个模块分别进行生成和管理,
并介绍了评估指标计算方法,最后介绍了评估系统在实现中的具体技术手段和模块化优势。

第五章为实验和分析。
本章针对静态分析评估系统的目标需求,设计了针对性的实验,
使用clang-tidy作为实验静态分析工具,将其规则进行细致评估和分析,
呈现系统的效果和多工具可拓展性。
最后基于实验结果\tc~静态分析工具评估问题进行了优势和不足分析。
