% !TeX root = ../main.tex
\chapter{结论与展望}

\section{设计初衷与实现成果}
\section{应用成效分析}
\section{系统存在的不足}
\section{展望与改进方向}


\iffalse
	第1章绪论 .................................................................................. 1
	1.1研究背景与意义 ...................................................................... 1
	1.1.1C/C++ 静态分析工具及其应用现状 ........................................ 2
	1.1.2静态分析工具评估的国内外研究现状 ..................................... 3
	1.1.3研究意义 ......................................................................... 5
	1.2本文主要工作内容 ................................................................... 6
	1.3本文工作的创新性及主要技术特色 .............................................. 6
	1.4论文组织与章节结构安排 .......................................................... 6
	第2章论文相关技术 ...................................................................... 7
	2.1静态分析工具及其原理 ............................................................. 7
	2.2静态分析工具评估方式 ............................................................. 7
	2.2.1测试集 ............................................................................ 7
	2.2.2评估指标 ......................................................................... 8
	2.3代码生成 .............................................................................. 10
	2.4容器化 ................................................................................. 10
	第3章场景化覆盖能力评估问题建模 ................................................ 11
	3.1静态分析工具的原理和对应的能力边界 ....................................... 11
	3.1.1数据传递检查能力和边界 ................................................... 11
	3.1.2路径爆炸检查能力和边界 ................................................... 11
	3.1.3指针问题检查能力和边界 ................................................... 11
	3.2场景空间的结构化建模 ............................................................ 11
	3.3场景驱动评估方法的优势与局限 ................................................ 11
	第4章面向场景的评估系统概要设计 ................................................ 12
	4.1系统总体架构与评估流程 ......................................................... 12
	4.2场景资产数据库 ..................................................................... 12
	4.3场景工厂 .............................................................................. 12
	4.4场景资产管理 ........................................................................ 12
	评估指标计算 .................................................................. 12
	4.4.1
	第5章面向场景的评估系统实现 ...................................................... 13
	5.1场景生成机制实现 .................................................................. 13
	5.2场景去重、分类与管理实现 ...................................................... 13
	5.3评估流程自动化实现 ............................................................... 13
	5.4系统可扩展性与替换性分析 ...................................................... 13
	第6章实验与分析
	第7章结论与展望

	前两章我写着写着觉得不应该在1.1.1写太多内容,或者直接将它放到2.1,
	但是3.1我又会写一边,不知道怎么办。
	我的重点是静态分析工具评估,
	而静态分析工具的原理技术很多,放在背景和意义中会臃肿,怎么办?

	另外如果不在第一章写工具原理,
	这一段作为研究背景和意义是否足够?

	\section{研究背景与意义}
	% 软件安全很重要,\tc~在软件中的占比大
	软件在应用领域使用广泛、发展迅猛,
	软件的安全可信一直被研究人员高度重视。
	缺陷(Weakness)指软件或程序中存在的某种破坏正常运行能力的问题、错误。
	上至国家安全,下至百姓生活,
	软件的缺陷都意味着或大或小的损失。
	在软件不断发展的过程中,
	减少软件缺陷是一个重大的课题。
	在软件语言家族中,\tc~占据比较大的比例,
	并且这类软件往往承担重要的低层功能,
	是构建操作系统、大型游戏引擎、数据库、
	嵌入式系统等对性能、效率和底层硬件控制有极致要求的软件的基石。
	\tc~软件的安全可信至关重要。

	% 静态分析是保障软件安全的重要一环
	为了减少软件缺陷,静态分析技术被开发人员广泛使用。
	静态分析是指通过静态地检查代码本身而非运行程序,
	来分析软件的性质的过程,静态分析技术可以应用在软件开发的各个阶段。
	静态分析技术依赖形式化的原理,可以实现快速分析,
	能够在分析精度和速度之间找到平衡。
	另外,由于软件缺陷发现的越早,造成的成本越低,
	而静态分析技术能在软件开发多生命周期发挥作用,
	所以静态分析技术能够尽早地发现并帮助修复软件缺陷,
	以降低缺陷造成的开发成本和实质安全成本。
	由于速度快、不需要实际运行等优势,
	静态分析技术成为了\tc~软件的缺陷检测的常用工具,
	显著控制了软件缺陷。

	% 提高静态分析质量和效率,能够帮助保障软件安全
	在追求安全性和可靠性的大环境下,
	软件行业对清楚的了解静态分析工具的分析精度提出了更高的要求。
	根据静态分析的莱斯定理(Rice's Theorem)\cite{riceTheory},
	静态分析工具永远不能保证检测出软件的所有缺陷。
	对于软件缺陷的保障来说,
	高效充分评估静态分析工具的检测能力,
	是使用静态分析工具时的迫切需求,
	是保障软件质量的重要环节。
	测试人员需要深入了解工具的实际检测能力,
	了解工具对具体程序问题的覆盖能力,
	才能在面对大规模代码的静态分析中,
	实现更加精准和高效的质量保障。

	% 静态分析工具评估的重要性
	在提升静态分析工具分析精度的道路上,
	准确清晰地评估静态分析工具的能力现状也相应的变得非常重要。
	每个缺陷都对应着众多场景,
	\textbf{场景}(Scene)描述了缺陷发生的实际条件和路径。
	测试人员希望能够选择一个或一套精确的静态分析工具,
	在开发的过程中实质上减少软件的缺陷,
	可以尽可能多的覆盖更多的缺陷场景。
	测试人员还希望能够直观的看到静态分析工具的能力,
	这样不仅能够指导软件缺陷检查工作,
	也对静态分析工具的研究人员提供准确的优化目标,
	更能向外界提供足够的材料支撑,
	说明自身软件的安全性。


	或者,这个版本应不应该继续写?

	\subsection{研究意义}

	基于前文对静态分析工具和评估的背景介绍,
	静态分析工具的评估对软件安全有长远而重要的意义。
	本文旨在基于前人对静态分析工具评估研究的基础上,
	将该问题进一步系统化,并结合当前的前沿技术加以实践,
	设计并实现一套\tc~静态分析工具自动化评估系统。

	具体来说,本文的研究意义主要在以下几方面。

	todo
	\iffalse
		\begin{enumerate}
			\item 用系统化的思路对静态分析工具的评估测试用例拓展问题进行建模,
			      对场景这个概念进行更系统和清晰的定义,
			      将缺陷和场景的关系进一步讨论,
			      设计出适合静态分析工具的
			      让测试集的拓展可以更准确的覆盖到更广的缺陷场景
			\item todo
		\end{enumerate}
	\fi
\fi
