% !TeX root = ../main.tex


% EvaluateModel
% → 定义问题本身,给出评估的数学/结构化对象

\chapter{场景化覆盖能力评估问题建模}

本章针对静态分析工具的场景化覆盖能力评估进行抽象化讨论,
旨在明确基于场景空间的静态分析工具能力评估方法应该如何设计。

从前文可以得出,静态分析工具的评估重点在于测试集的设计和管理,
有了组织清晰的测试集,
实际测试的结果才可以清晰的展示静态分析工具的能力地图。
那么应该怎样设计测试集空间呢?
JTS给出了一些参考,
它使用Functional维度和FLow维度组织测试集,
这两个正交维度在每一个缺陷下都各自展开,
但是这样的维度仅能触及工具的一部分能力,
不能对其能力进行更细粒度的检查。

静态分析工具的检测能力不仅与自身原理和具体规则相关,
还与程序场景的结构特征高度相关。
为了对评估问题进行建模,
有必要从工具原理出发,
结合场景的代码结构,
抽象其在不同场景下的能力边界。
因此,本文对静态分析工具的检查能力进行细致分析,
并基于此设计场景空间维度。

本文的分析目标不以工具为单位,
而以工具的每一个规则为单位。
理由是工具不是单一能力,
每个规则都对应着不同的分析策略和能力边界,
要想对工具进行完整的评估,
必须以规则为粒度,以分析范围为限制,以能力边界为场景维度。

\section{静态分析工具的能力边界}

本文把静态分析工具的能力分为两类,
一类是工具所能分析的范围,
具体来说,包括静态分析工具规则所对应的缺陷范围,
以及规则所能检查的工程作用域。

第二类是基于原理的能力范围,
分为语义建模深度、
路径敏感程度、
路径状态空间控制程度、
约束精细化能力。

\subsection{分析缺陷范围边界}

每一个工具规则都对应着若干缺陷。
我们能够将每一个场景测例划分到唯一的缺陷中,
但不能把每一个规则划分到唯一的缺陷中,
规则对应的是一个缺陷子集,
该子集中的所有的测例都应该作为规则评估的测试对象。

例如Clang-tidy有一条规则,
名为

\iffalse
	在固定缺陷类型(或缺陷族)的前提下,评估工具在不同能力边界维度上的表现。
	缺陷类型决定了分析所需的最低语义能力,因此本文将缺陷类型作为评估前提,而非能力维度本身。
	场景要诱导静态分析工具错误理解
\fi

\begin{enumerate}
	从工具的检查能力出发,将场景空间分为以下维度,
	\item 缺陷:检查的是哪个缺陷 :约束条件/实验前提
	      不同缺陷对分析能力的要求完全不同:
	      空指针 → 路径敏感
	      缓冲区溢出 → 数值域 + 路径
	      API 误用 → 时序/状态机

	\item 分析作用域:在什么样的工程维度内分析?
	      刻度:
	      单函数
	      跨函数
	      上下文敏感跨函数
	      跨文件
	      全程序

	\item 语义建模深度:能理解到什么语义层、是否建模程序状态
	      刻度:无语义 → 语法级 → 控制流级 → 状态级

	      | 能力刻度 | 本质     |
	      | ---- | ------ |
	      | 无语义  | 文本匹配   |
	      | 语法级  | 只看结构   |
	      | 控制流级 | 理解执行顺序 |
	      | 状态级  | 跟踪变量取值 |

	\item 路径敏感程度:能区分多少执行路径?
	      刻度:
	      路径不敏感
	      分支敏感
	      循环敏感
	      局部路径敏感
	      完整的路径敏感

	\item 路径状态空间控制:如何抑制路径爆炸?能不能走得动复杂路径?
	      刻度:
	      无抑制
	      主动合并状态
	      限制路径深度
	      启用widen
	      {
	      widen:
	      当发现状态在“单调变化”时,
	      直接跳到一个“稳定上界”。

	      int i = 0;
	      while (i < 1000) {
			      i++;
		      }
	      use(i);

	      无 widen:路径/状态爆炸
	      有 widen:i >= 0,分析快速收敛
	      }
	      有非平凡循环处理能力
	      {
	      max_loop_num 属于 工程控制参数
	      “是否能处理非平凡循环” 属于 能力边界判断
	      }


	\item 约束精细化能力:是否验证路径可达性,仅做是否判断
	      刻度:
	      不约束
	      约束
\end{enumerate}

\section{场景空间的结构化建模}

为了系统化刻画影响静态分析工具检测能力的因素,本文引入场景空间的概念……

\section{场景空间驱动的评估方法的优势与局限}
场景驱动评估并非替代传统测试集方法,而是提供了一种从能力视角审视评估问题的补充路径。

% \subsection{二级节标题}

% \subsubsection{三级节标题}

% \section{脚注}

% Lorem ipsum dolor sit amet, consectetur adipiscing elit, sed do eiusmod tempor
% incididunt ut labore et dolore magna aliqua.
% \footnote{Ut enim ad minim veniam, quis nostrud exercitation ullamco laboris
% 	nisi ut aliquip ex ea commodo consequat.
% 	Duis aute irure dolor in reprehenderit in voluptate velit esse cillum dolore
% 	eu fugiat nulla pariatur.}
