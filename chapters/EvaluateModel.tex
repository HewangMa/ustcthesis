% !TeX root = ../main.tex


% EvaluateModel
% → 定义问题本身,给出评估的数学/结构化对象

\chapter{覆盖能力评估问题建模}
\label{sec:覆盖能力评估问题建模}

本章针对静态分析工具的覆盖能力评估进行抽象化讨论,
旨在明确基于场景空间的静态分析工具能力评估方法应该如何设计。

从前文可以得出,
静态分析工具的评估重点在于测试集的设计和管理,
有了组织清晰的测试集,
实际测试的结果才可以清晰的展示静态分析工具的能力地图。
那么应该怎样设计测试集空间呢?
在章节\ref{sec:测试集}中介绍的JTS给出了一些参考,
它使用Functional类型和FLow类型这两个正交维度,
在每一个缺陷下都各自展开,
但是这样的维度仅能触及工具的一部分能力,
不能对其能力进行更细粒度的检查。

静态分析工具的检测能力不仅与自身原理和具体规则相关,
还与程序场景的结构特征高度相关。
为了对评估问题进行建模,
有必要从工具原理出发,
结合场景的代码结构,
抽象其在不同场景下的能力边界。
因此,本文对静态分析工具的检查能力进行细致分析,
并基于此设计场景空间维度。

% 分析以规则为单位
本文的分析目标不以工具为单位,
而以工具的每一个规则为单位。
理由是工具不是单一能力,
工具的每个规则都对应着不同的分析策略和能力边界。
下文以规则一词指代静态分析工具的规则。
因此,
本文在后续建模与评估过程中,
统一以规则为最小评估单元,
将规则能力的覆盖情况作为工具整体能力评估的基础,
并且以能力边界为维度组织场景空间。

\section{静态分析工具的能力边界}
\label{sec:静态分析工具的能力边界}

本文把静态分析工具的能力分为两类,
一类是工具所能分析的范围,
具体来说,
包括静态分析工具规则所对应的缺陷范围,
和规则所能检查的工程作用域。
第二类是理解程序的能力范围,
分为语义建模深度、
路径敏感程度、
路径状态空间控制程度、
约束精化能力、
并发语义建模能力。

\subsection{分析缺陷范围边界}
\label{sec:分析缺陷范围边界}

工具的每一个规则都对应着若干缺陷。
每一个规则也都有着不同的能力要求,
例如分析空指针的规则需要对路径敏感,
而分析缓冲区的规则则需要对数值域敏感。
我们能够将每一个场景测例划分到唯一的缺陷中,
但不能把每一个规则划分到唯一的缺陷中,
规则对应的是一个缺陷子集,
该子集中的所有的测例都应该作为规则评估的测试对象。

例如Clang-tidy有一条规则,
名为\texttt{bugprone-unused-return-value},
功能是检查没有使用的函数返回值,
它对应着两个缺陷,分别是
\texttt{CWE-252:Unchecked\_Return\_Value}
未检查函数返回值,和
\texttt{CWE-253:Incorrect\_Check\_of\_Function\_Return\_Value}
检查函数返回值不当。

再例如Clang-tidy的一条规则,
名为\texttt{bugprone-empty-catch},
功能是检查并建议解决空的catch语句,
对应着两个缺陷,分别是
\texttt{CWE-390:Error\_Without\_Action}
产生了error但不执行任何动作,和
\texttt{CWE-391:Unchecked\_Error\_Condition}
忽略异常和其他错误条件。

以上两个规则在理想状态下,
都应该覆盖对应的缺陷集中的所有场景测例。
而相应的,
不划分在对应缺陷集中的任何场景测例都不需要被该规则负责,
要么它们由工具中的其他规则负责,
要么他们不在工具设计之初希望覆盖的范围之内。
图\ref{fig:工具-规则-缺陷集}解释了这种关系。

\begin{figure}[ht]
	\centering
	\includegraphics[width=0.7\textwidth]{工具-规则-缺陷集.png}
	\caption{工具、规则、缺陷集之间的关系}
	\label{fig:工具-规则-缺陷集}
\end{figure}

\subsection{分析作用域边界}

要评估工具的能力,
工具规则所能分析的作用域边界也是重要的限制,
即工具规则能在什么工程范围内分析?

在章节\ref{sec:\tc~静态分析工具及其原理}中提到,
静态分析工具在跨函数、跨文件等情况下的分析能力参差不齐。
在典型的数据流分析中,
一个规则也许可以分析在一个函数内的关键数据,
但通过多种返回值传到函数外的数据就不能分析了;
一个规则也许可以分析跨函数数据,
但有上下文不同的跨函数数据就不能分析了;
一个规则也许可以分析上下文敏感的数据,
但跨文件传递的数据就不能分析了;
一个规则也许可以分析跨文件数据,
但通过文件读写、环境配置等情况的数据就不能分析了。

在非典型数据流分析的缺陷中,
也可以考虑规则的分析作用域。
比如前文提到的缺陷
\texttt{CWE-390:Error\_Without\_Action}
在error发生的位置,
也许程序确实没有处理动作,
但是该error被另一个文件的函数中的动作处理了,
那么对于分析域不同的工具规则来说,
也许一个规则会告警,
产生误报,
而另一个规则则不会告警,
其误报率则较低。
基于以上的分析和举例,
分析作用域边界应当作为一个规则能力的维度。

设定作用域为一个维度后,
应当考虑的是如何设计刻度。
刻度指的是规则在这个维度上的能力节点。
本文将分析作用域这个维度下的刻度设置如下:
\begin{enumerate}
	\item 单函数。
	      这是一个最小能力的规则应该能处理的分析范围。

	\item 上下文无关跨函数。
	      这个刻度下的规则可以处理经过函数传递的数据相关的分析,
	      但该刻度下存在一个子维度,即能够分析数据经过多少层函数传递。

	\item 上下文敏感跨函数。
	      这个刻度下的规则可以处理与上下文有关的跨函数数据分析,
	      例如全局变量、静态变量等。

	\item 跨文件。
	      这个刻度下的规则可以处理跨文件的数据分析,
	      例如一个文件中设计了某个功能,
	      通过一个接口传递给另一个文件中的函数。

	\item 全程序。
	      这个刻度下的规则可以分析全程序的数据流,
	      例如一个功能从配置文件中读取数据,
	      与它相关的数据流分析也可以做到。
\end{enumerate}

\subsection{语义建模深度边界}

在章节\ref{sec:\tc~静态分析工具及其原理}中提到,
不同工具在系统建模阶段对程序语义的抽象程度不同,
例如,对于Flawfinder工具来说,
它的建模方式是基于字符串的,
也就意味着它会被一些程序语义、语法、状态欺骗。
而Clang-tidy是基于编译构造的AST的,
它比理解字符串高一层,
可以理解程序的结构关系,
但是对变量、程序的状态所知不多。
为此,本文将语义建模深度作为一个工具能力的边界,
这个维度并不是简单的对工具系统建模属性的描述,
而是划分工具理解程序的能力边界,
是工具对程序进行抽象的能力。
该维度可以使用场景测例对能力边界进行探索和评估。

本文将该维度下的刻度划分设置如下。
需要说明的是,
以下刻度并非严格的能力等级划分,
不同工具可能在不同刻度上呈现交叉特性。

\begin{enumerate}
	\item 无语义。
	      这个刻度下的工具规则不能理解任何程序语义,
	      仅从字符串和模式出发进行匹配分析。

	\item 语法级。
	      这个刻度下的工具规则可以理解if、while等语法,
	      了解程序的结构,
	      但对程序如何执行以及其中的变化和限制无法分析。

	\item 控制流级。
	      这个刻度下的工具规则可以理解程序的执行流程,
	      也就是理解了CFG控制流图。
	      可以分析一段程序有可能从哪些程序基本块进入,
	      到哪些基本块,
	      但不必区分不同执行路径。

	\item 状态级。
	      这个刻度下的规则不仅可以根据CFG理解程序的执行流程,
	      还能基于数据理解程序在某个时刻的状态,
	      也就是有数据流分析能力,
	      可以跟踪变量的取值。
\end{enumerate}

语义建模深度这个维度的划分粒度大约是工具级别的,
因为同一个工具的系统建模属性一般是一致的,
大部分规则都在同一种系统建模中执行检查。
在评估的时候也许这个维度的数据相对集中,
但不排除在这个维度上不同的规则有不同的表现,
所以依然以工具规则为粒度进行评估。

\subsection{路径敏感程度边界}

本文在在章节\ref{sec:\tc~静态分析工具及其原理}中提到,
路径敏感是工具检测过程的一种。
路径敏感程度是一种和语义建模深度正交的维度。
语义建模深度表征的是工具用了什么样的模型,
是一种表达能力;
而路径敏感则表征工具如何使用这个模型,
是否区分不同执行路径,
是一种分析精度策略。
为此,路径敏感程度和语义建模深度维度是正交的。
正交不意味着在该二维矩阵中的每个位置都一定存在场景,
也不意味着这两个维度的数据是无相关性的,
只是意味着这两个维度表征的是不同的能力。

本文将路径敏感程度维度的刻度设置如下。

\begin{enumerate}
	\item 路径不敏感。
	      这个刻度下的工具规则不能区分任何路径。

	\item 分支敏感。
	      这个刻度下的工具规则可以在遇到if/else语句的时候区分路径。

	\item 循环敏感。
	      这个刻度下的工具规则可以在遇到for/while循环的边界、
	      break/continue等分支时区分路径。

	\item 几乎完整的路径敏感。
	      除了若干深层路径,
	      这个刻度下的工具规则可以探知程序的大部分执行路径。
	      由于基于路径敏感分析的极高代价,
	      对静态分析工具来说,
	      在实践工程中面对大规模程序,
	      几乎完整的路径敏感是比较现实的考虑。

	\item 完整的路径敏感。
	      相比与前一个刻度,
	      这个刻度下的工具规则可以探知程序的每个可能执行路径,
	      但这只在理论上存在,
	      实际的工具不可能达到这个刻度。
\end{enumerate}

\subsection{路径状态空间控制能力边界}

路径爆炸是基于路径和状态的静态分析工具都需要面对的。
程序复杂度提高时,
路径数量并不是线性增长,
而是呈指数级增多,
并且在遇到循环等结构时,
路径数有可能是无限的,
故称为路径爆炸。
对路径状态空间的控制是静态分析工具的重要能力,
所评估工具能不能抑制路径爆炸?
能不能分析的了复杂路径?
为此,本文将路径状态空间控制能力作为一个边界。

控制路径爆炸的能力和语义建模深度、路径敏感程度都是正交的,
首先,这个维度的能力不管有没有,
都不影响前面介绍的两种能力。
虽然路径抑制能力的前提是有对路径敏感,
但这只会带来一个类似上三角矩阵的场景空间,
不影响能力的正交。

本文为该边界设置如下标志刻度。

\begin{enumerate}
	\item 无抑制。
	      该刻度下的工具规则没有任何抑制路径爆炸的手段。

	\item 主动合并状态。
	      这个刻度下的工具规则可以主动合并分支,
	      举例来说,
	      对于if和else分支的程序段逻辑完全相同,
	      上一个刻度的工具也许会分为两个路径分析,
	      但本刻度下的工具会将其作为一个路径分析,
	      这样一来就抑制了路径的倍数,
	      如果内层路径数量本身就很多,
	      那么两倍的分析量也足以让工具负担不起。

	\item 限制路径深度。
	      面对有可能无限增长的路径和状态,
	      静态分析工具需要有工程兜底能力,
	      是否设置了路径深度的上限。
	      在这个刻度下也有一个子维度,
	      即面对在非平凡循环时的展开限制。
	      正如在章节\ref{sec:\tc~静态分析工具及其原理}中提到,
	      MLN是一个工具在面对循环时展开的层数,
	      值一般不会超过5,
	      即工具会将循环展开MLN层进行分析,
	      如果程序在大于MLN层中有非平凡表现,
	      则工具识别不到,
	      比如代码\ref{lst:非平凡循环}中,
	      MLN值大于5与小于5的工具对其分析结果是不同的。
	      该参数是路径敏感的工具一定会有的,
	      否则很容易遇到无限路径的情况,
	      对于超过MLN的循环部分,
	      要么截断,
	      要么使用下文的widen技术。
	      为此,限制路径深度,
	      是工具规则在路径状态空间控制能力边界中的重要刻度。

	\item 启用widen。
	      widen 是抽象解释中的一种状态加速技术,
	      用于在循环或递归分析中,
	      通过引入更粗粒度的抽象状态,
	      使状态序列在有限步内收敛,
	      从而保证分析终止。
	      典型的应用场景是单调的循环中,
	      使用末端的循环子去概括程序状态,
	      比如代码\ref{lst:单调循环造成的路径增长}中,
	      无 widen 能力的工具会面对路径、状态爆炸,
	      而有 widen 能力的工具会使用i >= 0分析快速收敛。
	      widen技术是目前静态分析工具在路径空间控制中比较前沿的技术,
	      是否能利用widen是一个重要的能力刻度。
	      值得一提的是,
	      路径数量控制并没有一个完美的能力刻度。
\end{enumerate}

\subsection{约束精化能力}

约束精化指的是在静态分析过程中,
通过引入或加强路径约束,
判断某条分析得到的错误路径在真实程序中是否可达,
从而消除不可达路径导致的误报。
在程序的路径中,
有的路径是语法上可达,
但是语义上不可达的。

举例来说,
代码\ref{lst:语法可达但语义不可达}中有不可达路径,
在分析过程中要对这条路径进行处理时,
如果有约束精化,
则不必付出太大代价去计算,
也不会发出很多误报;
如果没有则反之。
这个维度涉及到非常多的精化策略,
为简化能力边界的建模,
考虑到评估系统的复杂度和可验证性,
本文将约束精化能力的刻度设置为有和没有,
在实验章节进行是和否的验证。

\subsection{并发语义建模能力}

并发导致的问题在大型程序中非常常见,
这也是静态分析工具难以处理的点。
这个维度的能力涉及到线程间执行交错、
线程之间共享状态、
同步原语这三个本质问题的建模能力,
并不是一般意义上对结构的建模就可以处理的。

在这个维度下的刻度可以设置为:
不支持并发(单线程假设)、
线程感知(识别线程边界)、
同步感知(理解锁/原子)、
交错建模(分析线程交错),
但考虑到该维度能力的系统可验证性,
仅做抽象建模,
不做实验验证。

总结本节,
本文将静态分析工具的能力边界分为以下维度。

\begin{enumerate}
	\item 规则所对应的缺陷范围。
	\item 规则所能检查的工程作用域。
	\item 语义建模深度。
	\item 路径敏感程度。
	\item 路径状态空间控制程度。
	\item 约束精化能力。
	\item 并发语义建模能力。
\end{enumerate}

其中前两个维度是工具所能分析的范围,
后面的维度是工具理解程序的能力范围,
本文对每个维度都设置了对应的能力刻度,
并且维度之间相互正交,
任何一个工具都一定有这七个维度的能力边界。
本文并非从单一测试集或单一指标出发评估静态分析工具,
而是尝试以能力边界为核心,
将工具的检测能力刻画为多维空间中的边界问题,
为后续评估方法的设计奠定理论基础。

\section{场景空间驱动的评估方法}

\subsection{场景空间}
\label{sec:场景空间}

本文关注的并非单个测试用例,
而是如何通过系统化的场景组织方式,
刻画静态分析工具在不同能力维度上的检测边界。
为了系统化刻画影响静态分析工具检测能力的因素,
本文引入“场景空间”的概念如定义(\ref{eq:场景空间})。
需要强调的是,
本文所定义的场景空间并非严格意义上的数学空间,
而是一种用于组织和刻画测试场景的概念空间。

\begin{equation}
	\mathcal{S} = \langle D, P, S, C, L, R \rangle
	\label{eq:场景空间}
\end{equation}

其中的维度对应着章节\ref{sec:静态分析工具的能力边界}中的能力:

\begin{itemize}
	\item $D$ (Defect Set):缺陷集。
	      对应规则所对应的缺陷范围,
	      该维度的不同场景有着不同的缺陷内容。
	\item $P$ (Program Span):程序跨度。
	      对应规则所能检查的工程作用域,
	      该维度的不同场景拥有不同的程序跨度。
	\item $S$ (Semantic Complexity):语义建模复杂度。
	      对应能力边界中的语义建模深度,
	      该维度的不同场景需要不同的语义建模能力。
	\item $C$ (Path Complexity):路径结构复杂度。
	      对应能力边界中的路径敏感程度,
	      关注的是路径分支结构的复杂性。
	      该维度的不同场景拥有不同的路径结构,
	      尝试考验工具规则的对不同结构的路径的识别能力。
	\item $L$ (Path Depth):路径深度与状态空间控制。
	      对应能力边界中的路径状态空间控制程度。
	      关注的是路径展开后的状态空间规模。
	      该维度的不同场景拥有不同的路径深度和复杂度,
	      尝试考验工具规则的路径爆炸控制能力。
	\item $R$ (Reachability):有无不可达路径。
	      对应能力边界中的约束精化与可达性判定能力。
	      该维度的不同场景都有着不可达路径,
	      基于对该维度场景的检测结果,
	      可以评估工具规则是否有约束精化能力。
\end{itemize}

在该场景空间并不像数学的空间那样连续且有序,
而是在设计好的刻度下有着不同的位置。
在此场景空间的基础上,
本文将测试集视为场景空间上的一组有限采样点。
对于一个测试用例,
都可以使用本空间定义,
在缺陷发生的路径上或者相关的环境中,
在对应的维度上都找到对应的位置。

需要强调的是,
本文所定义的场景空间并非要求每一个缺陷实例在所有维度上均具备有效取值。
对于不同类型的缺陷,
其所涉及的程序结构复杂度与分析难度存在显著差异,
因此其可刻画的维度集合亦不完全一致。
场景维度是一种可选属性,
仅在缺陷语义或分析难度需要时才进行刻画。

例如,
对于资源泄漏类缺陷,
往往涉及复杂的路径结构、
较深的状态空间展开以及可达性判定问题,
因此在 $D,P,S,C,L,R$ 等多个维度上均具有明确刻画;
而对于缺陷 \texttt{CWE-665:Improper\_Initialization},
其问题通常发生在局部作用域内,
不涉及复杂路径分支或跨过程语义,
在场景空间中主要体现
除了缺陷类型 $D$之外,
只有有限的程序跨度 $P$ 维度刻画,
其余维度在该类缺陷中不具有显著区分意义。

\subsection{测试集拓展方法}

有了场景空间的基础,
接下来介绍本文的测试集扩展方法,
也就是在这个场景空间中,
增加场景测例。
% 场景要诱导静态分析工具错误理解。

目前主要的一些补充用例的方向可以分为两类,
正向梳理和被动补充,
如图\ref{fig:测试集拓展方式}。

\begin{figure}[ht]
	\centering
	\includegraphics[width=\textwidth]{测试集拓展方式.png}
	\caption{拓展测试集的两类方式}
	\label{fig:测试集拓展方式}
\end{figure}

% 正向梳理
正向梳理指的是基于现有测试集和代码的特征等已有知识,
通过调整结构、变化传递方式等正向手段增加测试用例,
相当于在一个多维的世界中进行深度搜索,
正向梳理对应于沿一个或多个能力维度进行系统性的扩展,
通过结构变化、语义扰动和路径控制,
加上适当的剪支和限定,
可以实现构建一个接近“全覆盖”的场景测试集,
在既定刻度范围内探索能力边界。

% 被动补充
另一方面,被动补充方法指的是从实际触发的软件缺陷中,
提取相应的缺陷场景和正确场景。
这需要从缺陷管理系统(JIRA、GitHub 等)
和信息披露系统收集开源软件的缺陷信息,
进行爬取和复杂的数据整理。
这可以让缺陷更加适应实际软件缺陷,
让静态分析工具的评估更接近真实的应用场景。

% JTS的关系
本文在章节\ref{sec:测试集}提到,
JTS使用Functional类型和Flow类型作为测例文件名标记,
但是这只是一种测试用例的组织方式。
JTS的Functional类型可以看作本文的场景空间中程序跨度维度的散点,
Flow类型可以看作场景空间中路径复杂度的散点。
JTS在其他维度上也有所体现,
但是属于扁平的、零星的体现,
并不能系统化的评估出静态分析工具的多维度能力。
在本文的场景空间中,
JTS的每个测试用例都可以被看作一个空间上的元素,
在评估体系的初始阶段,
本文使用被动补充的方式,
将JTS的每个测例都映射到场景空间中,
让场景空间的空缺位置有相关的参考,
为后续正向梳理提供基础。

\subsection{基于场景空间的测试结果解析}
\label{sec:基于场景空间的测试结果解析}

在一定规模的场景空间的基础上,
接下来介绍本文的测试结果解析方法。

对于每一个测试用例,
都有在场景空间的位置,
在某个维度的两个刻度之间,
可能有多个相近测试用例,
比如路径深度的差别等等。
参考JTS的测例组织方式,
本文在一个场景空间位置上的一个测例都包含两份代码,
bad和good,
分别代表有缺陷的场景和对应的无缺陷安全场景。

在评估标准上,
本文使用两套体系,
一方面使用比例算法,
根据评估对象在对应的缺陷集中所有测例下的数据表现,
整理成表\ref{tab:静态分析检查得到的数据}的数据,
计算出精准度、召回率、F1分数等指标,
用以从宏观维度上评估工具规则的定性指标;
另一方面,
对于场景空间的每个维度,
将测例投影到这个维度上,
考察工具规则在这个维度上的表现,
对应的评估工具规则在这个能力方向上的边界。
这种投影式的分析方法,
使得工具在不同能力维度上的表现可以被分别观察,
避免了单一指标掩盖能力短板的问题。

一方面,
对于宏观指标,
关键在于准确区分测试用例的真实值,
即区分对于一个缺陷来说,
哪些应该告警,
哪些不应该告警
区分方法如图\ref{fig:测例真实值}所示。
由于高误报是静态分析工具的一大缺点,
必须有准确的数据衡量这一点。
对于不应该告警的测例,
如果告警多次,
需要计算多次误报。

\begin{figure}[ht]
	\centering
	\includegraphics[width=0.9\textwidth]{测例真实值.png}
	\caption{测例真实值}
	\label{fig:测例真实值}
\end{figure}

另一方面,
对于基于边界的工具能力评估,
本文按照能力维度划分,
对于选定的维度,
将目标规则对应的应覆盖场景投影到该维度上,
如果有足够的场景并不全都落在刻度为0的位置,
则说明该维度有可用的评估场景。
一般来说,
经过投影和覆盖标记,
四种覆盖分布如图\ref{fig:覆盖分布情况}所示,

\begin{figure}[ht]
	\centering
	\includegraphics[width=\textwidth]{覆盖分布情况.png}
	\caption{覆盖分布情况}
	\label{fig:覆盖分布情况}
\end{figure}

对于单调的维度,
如 $P$ 程序跨度,
理想中的分布应当如图\ref{fig:覆盖分布情况}中的(a)所示,
则虚线位置即为规则在该维度上的能力边界;
对于非单调的维度,
如 $C$ 路径结构复杂度,
它并不是完全单调,
分支和循环结构很难说是单调的,
理想中的分布应当如图\ref{fig:覆盖分布情况}中的(b)所示,
则刻度0、2、3是规则可以覆盖的,
而刻度1、4是规则无法覆盖的;
如果某一维度的覆盖分布如图\ref{fig:覆盖分布情况}中的(c)所示,
则说明有其他程序特征在影响规则的检查结果,
该维度与工具规则的检查能力不相关,
规则在该维度上的能力边界适合被评估;
实际情况中,
某一维度的覆盖分布可能如图\ref{fig:覆盖分布情况}中的(d)所示,
只有零散的几个不符合某刻度的总体情况,
可以用百分比来衡量规则在该刻度上的能力,
本文将某刻度上的覆盖比例记为mark\_coverage,
其计算公式如公式(\ref{eq:刻度覆盖率})所示,

\begin{equation}
	\text{mark\_coverage} = \frac{covered\_cnt}{mark\_cnt}
	\label{eq:刻度覆盖率}
\end{equation}

然后将这些刻度覆盖率按照刻度排序,
可以展现该维度的能力现状。
如果某规则的所有dimensional\_coverage的方差不大,
即分布比较集中,
则说明该规则在该维度上的能力区分并不大,
符合图\ref{fig:覆盖分布情况}中(d)图所示情况;
如果某规则的所有dimensional\_coverage的方差较大,
即分布比较分散,
则说明该规则在该维度上的能力有明显区分,
符合图\ref{fig:覆盖分布情况}中(d)图甚至(a)、(b)图所示情况,
这样的刻度覆盖率可以作为清晰的能力指标。

综上所述,
基于占比的宏观指标和基于能力边界的评估共同构成本文的评估方法。

本节介绍了场景空间驱动的评估方法,
通过场景空间定义、
场景空间补充方法、
评估指标设置三个部分介绍了本文的评估方法。
总的来说,
本文使用场景空间系统的定义了评估范式,
既可以计算各种宏观指标,
也可以探索工具的能力边界。

\section{场景空间驱动的评估方法的优势与局限}
% todo 这个section 是 ai 写的, 需要修改
场景空间驱动的评估方法并非旨在替代传统基于测试集和指标统计的评估方式,
而是从工具能力边界的角度,
为静态分析工具的评估提供一种结构化、可解释的补充路径。

其主要优势体现在以下几个方面。
首先,
通过引入场景空间的概念,
测试用例不再被视为彼此孤立的样本,
而是被组织在一个由能力维度刻画的统一框架中,
从而能够系统性地分析工具在不同能力方向上的表现。
这种组织方式有助于揭示传统宏观指标难以反映的能力短板,
例如在路径深度、约束精化等特定能力维度上的失效边界。

其次,
场景空间为测试集的扩展提供了明确的方向指引。
无论是基于已有知识进行的正向梳理,
还是从真实缺陷中抽取场景的被动补充,
都可以被理解为在场景空间中的有目的采样过程,
避免了测试用例扩展过程中的盲目性和随意性,
提高了测试集构建的系统性和可复用性。

此外,
该方法在评估结果分析阶段具备较强的可解释性。
通过将测试结果投影到不同能力维度上,
可以直观地观察工具规则在各能力方向上的有效范围,
从而为规则优化和工具改进提供针对性的参考。

与此同时,
场景空间驱动的评估方法也存在一定的局限性。
一方面,
场景空间的维度划分和刻度设计在一定程度上依赖于研究者的经验和领域知识,
不同的维度选择可能会影响评估结果的侧重点,
难以保证绝对的客观性和完备性。
另一方面,
随着场景空间维度和刻度数量的增加,
测试用例的构建成本和评估成本也会相应上升,
在实际应用中需要在评估精度和资源消耗之间进行权衡。

综上所述,
场景空间驱动的评估方法为静态分析工具的能力评估提供了一种新的视角,
能够在宏观指标之外刻画工具的能力边界。
其更适合作为一种补充性的评估范式,
与传统测试集方法相结合,
共同服务于静态分析工具的设计、比较和优化。

总结本章,
为了对\tc~静态分析工具进行更系统的评估,
本章提出了基于能力边界的工具评估方法,
以期完成更系统、更准确并更及时的工具评估。
后文评估系统的设计和实现都依赖于本章对问题的分析和建模。