% !TeX root = ../main.tex


% EvaluateModel
% → 定义问题本身,给出评估的数学/结构化对象

\chapter{场景化覆盖能力评估问题建模}

本章针对静态分析工具的场景化覆盖能力评估进行抽象化讨论,
旨在明确基于场景空间的静态分析工具能力评估方法应该如何设计。

从前文可以得出,
静态分析工具的评估重点在于测试集的设计和管理,
有了组织清晰的测试集,
实际测试的结果才可以清晰的展示静态分析工具的能力地图。
那么应该怎样设计测试集空间呢?
JTS给出了一些参考,
它使用Functional维度和FLow维度组织测试集,
这两个正交维度在每一个缺陷下都各自展开,
但是这样的维度仅能触及工具的一部分能力,
不能对其能力进行更细粒度的检查。

静态分析工具的检测能力不仅与自身原理和具体规则相关,
还与程序场景的结构特征高度相关。
为了对评估问题进行建模,
有必要从工具原理出发,
结合场景的代码结构,
抽象其在不同场景下的能力边界。
因此,本文对静态分析工具的检查能力进行细致分析,
并基于此设计场景空间维度。

本文的分析目标不以工具为单位,
而以工具的每一个规则为单位。
理由是工具不是单一能力,
工具的每个规则都对应着不同的分析策略和能力边界。
下文以规则一词指代静态分析工具的规则。
所以要强调的是,
要想对工具进行完整的评估,
必须以规则为粒度,
以分析范围为限制,
以能力边界为场景维度。

\section{静态分析工具的能力边界}

本文把静态分析工具的能力分为两类,
一类是工具所能分析的范围,
具体来说,包括静态分析工具规则所对应的缺陷范围,
以及规则所能检查的工程作用域。

第二类是基于原理的能力范围,
分为语义建模深度、
路径敏感程度、
路径状态空间控制程度、
约束精细化能力。

\subsection{分析缺陷范围边界}

每一个工具规则都对应着若干缺陷。
我们能够将每一个场景测例划分到唯一的缺陷中,
但不能把每一个规则划分到唯一的缺陷中,
规则对应的是一个缺陷子集,
该子集中的所有的测例都应该作为规则评估的测试对象。

例如Clang-tidy有一条规则,
名为\texttt{bugprone-unused-return-value},
功能是检查没有使用的函数返回值,
它对应着两个缺陷,分别是
\texttt{CWE-252:Unchecked\_Return\_Value}
未检查函数返回值,和
\texttt{CWE-253:Incorrect\_Check\_of\_Function\_Return\_Value}
检查函数返回值不当。

再例如Clang-tidy的一条规则,
名为\texttt{bugprone-empty-catch},
功能是检查并建议解决空的catch语句,
对应着两个缺陷,分别是
\texttt{CWE-390:Error\_Without\_Action}
产生了error但不执行任何动作,和
\texttt{CWE-391:Unchecked\_Error\_Condition}
忽略异常和其他错误条件。

以上两个规则在理想状态下,
都应该覆盖对应的缺陷集中的所有场景测例。
而相应的,
不划分在对应缺陷集中的任何场景测例都不需要被该规则负责,
要么它们由工具中的其他规则负责,
要么他们不在工具设计之初希望覆盖的范围之内。
图示\ref{fig:工具-规则-缺陷集}解释了这种关系。

\begin{figure}[ht]
	\centering
	\includegraphics[width=0.8\textwidth]{工具-规则-缺陷集.png}
	\caption{工具、规则、缺陷集之间的关系}
	\label{fig:工具-规则-缺陷集}
\end{figure}

\subsection{分析作用域边界}

要评估工具的能力,
工具规则所能分析的作用域边界也是重要的限制,
即工具规则能在什么工程范围内分析?

在\ref{sec:\tc~静态分析工具及其原理}章节中提到,
静态分析工具在跨函数、跨文件等情况下的分析能力参差不齐。
在典型的数据流分析中,
一个规则也许可以分析在一个函数内的关键数据,
但通过多种返回值传到函数外的数据就不能分析了;
一个规则也许可以分析跨函数数据,
但有上下文不同的跨函数数据就不能分析了;
一个规则也许可以分析上下文敏感的数据,
但跨文件传递的数据就不能分析了;
一个规则也许可以分析跨文件数据,
但通过文件读写、环境配置等情况的数据就不能分析了。

在非典型数据流分析的缺陷中,
也可以考虑规则的分析作用域。
比如前文提到的缺陷
\texttt{CWE-390:Error\_Without\_Action}
在error发生的位置,
也许程序确实没有处理动作,
但是该error被另一个文件的函数中的动作处理了,
那么对于分析域不同的工具规则来说,
也许一个会告警,
产生误报,
而另一个则不会告警,
其误报率则较低。
基于以上的分析和举例,
分析作用域边界应当作为一个规则能力的维度。

设定作用域为一个维度后,
应当考虑的是如何设计刻度。
刻度指的是规则在这个维度上的能力节点。
本文将分析作用域这个维度下的刻度设置如下:
\begin{enumerate}
	\item 单函数。
	      这是一个最小能力的规则应该能处理的分析范围。
	\item 上下文无关跨函数。
	      这个刻度下的规则可以处理经过函数传递的数据相关的分析,
	      但该刻度下存在经过多少层函数传递依然是一个子维度。


\end{enumerate}


\iffalse
	在固定缺陷类型(或缺陷族)的前提下,评估工具在不同能力边界维度上的表现。
	缺陷类型决定了分析所需的最低语义能力,因此本文将缺陷类型作为评估前提,而非能力维度本身。
	场景要诱导静态分析工具错误理解、



	\begin{enumerate}
		\item 缺陷:检查的是哪个缺陷 :约束条件/实验前提
		      不同缺陷对分析能力的要求完全不同:
		      空指针 → 路径敏感
		      缓冲区溢出 → 数值域 + 路径
		      API 误用 → 时序/状态机

		\item 分析作用域:在什么样的工程维度内分析?
		      刻度:
		      单函数
		      跨函数(无上下文)
		      上下文敏感跨函数
		      跨文件
		      全程序

		\item 语义建模深度:能理解到什么语义层、是否建模程序状态
		      刻度:无语义 → 语法级 → 控制流级 → 状态级

		      | 能力刻度 | 本质     |
		      | ---- | ------ |
		      | 无语义  | 文本匹配   |
		      | 语法级  | 只看结构   |
		      | 控制流级 | 理解执行顺序 |
		      | 状态级  | 跟踪变量取值 |

		\item 路径敏感程度:能区分多少执行路径?
		      刻度:
		      路径不敏感
		      分支敏感
		      循环敏感
		      局部路径敏感
		      完整的路径敏感

		\item 路径状态空间控制:如何抑制路径爆炸?能不能走得动复杂路径?
		      刻度:
		      无抑制
		      主动合并状态
		      限制路径深度
		      启用widen
		      {
		      widen:
		      当发现状态在“单调变化”时,
		      直接跳到一个“稳定上界”。

		      int i = 0;
		      while (i < 1000) {
				      i++;
			      }
		      use(i);

		      无 widen:路径/状态爆炸
		      有 widen:i >= 0,分析快速收敛
		      }
		      有非平凡循环处理能力
		      {
		      \texttt{max\_loop\_num} 属于 工程控制参数
		      “是否能处理非平凡循环” 属于 能力边界判断
		      }


		\item 约束精细化能力:是否验证路径可达性,仅做是否判断
		      刻度:
		      不约束
		      约束
	\end{enumerate}
\fi

\section{场景空间的结构化建模}

为了系统化刻画影响静态分析工具检测能力的因素,本文引入场景空间的概念……

\section{场景空间驱动的评估方法的优势与局限}
场景驱动评估并非替代传统测试集方法,而是提供了一种从能力视角审视评估问题的补充路径。

% \subsection{二级节标题}

% \subsubsection{三级节标题}

% \section{脚注}

% Lorem ipsum dolor sit amet, consectetur adipiscing elit, sed do eiusmod tempor
% incididunt ut labore et dolore magna aliqua.
% \footnote{Ut enim ad minim veniam, quis nostrud exercitation ullamco laboris
% 	nisi ut aliquip ex ea commodo consequat.
% 	Duis aute irure dolor in reprehenderit in voluptate velit esse cillum dolore
% 	eu fugiat nulla pariatur.}
