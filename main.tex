% !TeX encoding = UTF-8
% !TeX program = xelatex
% !TeX spellcheck = en_US

\documentclass[degree=master,degree-type=professional]{ustcthesis}
% degree      = doctor | master | bachelor
% degree-type = academic | professional | engineering
% language    = chinese | english
% fontset     = windows | mac | ubuntu | fandol

% 加载宏包、全部的配置
\input{ustcsetup}


\begin{document}

\maketitle
\copyrightpage

\frontmatter
\include{chapters/innovations}  % 博士学位论文的创新性说明
% !TeX root = ../main.tex

\ustcsetup{
	keywords  = {C/C++静态分析工具,分析精度,缺陷场景,检索增强生成},
	keywords* = {C/C++ static analysis tools, analysis accuracy, defect scenarios , retrieval enhancement generation },
}

\begin{abstract}
	%% 开题报告版本

	% 本文针对企业在用C/C++静态分析工具在复杂代码场景中覆盖能力评估不足的问题,设计并实现一种面向具体规则、基于场景的静态分析工具分析精度评估系统。为解决传统评估方式依赖人工构造场景、覆盖面不足、难以量化等问题,本文深入研究了这类工具特性、关键的评估要点,设计引入“场景工厂”与“场景资产管理”两大核心模块,结合检索增强生成驱动的场景生成技术,构建具备演化能力的场景知识库,来实现工具分析精度的动态评估。

	% 本文的场景工厂采用检索增强生成和责任链设计模式,结合企业真实问题与通用静态分析规则,构造跨文件、深调用链等复杂代码场景,用于模拟静态分析器可能面临的挑战性代码结构;本文的场景资产管理支持对场景进行增删改查、分类和去重,该模块还包含量化计算方法,提供在不同规则维度下的精准度、召回率、F1 分数等指标计算功能;本文的场景知识库是系统的核心数据库,该数据库根据系统需求设计,提供评估所需关键信息,并支撑场景的使用和维护。

	% 考虑到本文目标系统实现主要针对企业内部专用的多种C/C++静态分析工具,所以,本文在测试验证环节,选用开源的clang-tidy  工具进行可验证实验,以展示本文的评估方法及所实现的目标系统的适用性、通用性和可用性。

	% 本文目标系统的应用,可以为各类面向C/C++语言的静态分析工具提供自动化评估测评功能,从而为这类静态分析工具的改进与优化提供决策依据。



	%% 导师版本 %% 

	% 企业在复杂代码场景中使用C/C++ 静态分析工具时,存在依赖人工构造场景、覆盖面不足、难以量化等问题。为解决传统评估、评测方式覆盖面及评估能力不足问题,本文深入研究了主流的C/C++ 静态分析工具的技术特性、评估的关键技术,设计引入了“场景工厂”与“场景资产管理”两大核心模块,结合检索增强生成驱动的场景生成技术,构建具备演化能力的场景知识库,以面向动态可变规则的方式,来实现C/C++静态分析工具的精度动态评估系统。这对于为这类静态分析工具的改进与优化提供更精准的决策依据具有重要意义。

	% 本文的主要工作和贡献包括:
	% \begin{enumerate}
	% 	\item 基于检索增强生成驱动的场景生成技术,构建具备演化能力的场景知识库,并基于该知识库,设计并实现了一种基于场景的静态分析工具分析精度评估系统。
	% 	\item 本文的场景工厂采用检索增强生成和责任链设计模式,结合企业真实问题与通用静态分析规则,构造跨文件、深调用链等复杂代码场景,用于模拟静态分析器可能面临的挑战性代码结构;其中,场景资产管理支持对场景进行增删改查、分类和去重,该模块还包含量化计算方法,提供在不同规则维度下的精准度、召回率、F1 分数等指标计算功能。而作为系统的核心数据库的场景知识库管理模块,可根据企业需求设计并支撑场景的使用和维护,为评估提供所需关键且充分的用例场景。
	% 	\item 考虑到本文目标系统实现主要针对企业内部专用的多种 C/C++ 静态分析工具,所以,本文在测试验证环节,选用开源的 clang-tidy 工具进行可验证实验,以展示本文的评估方法及所实现的目标系统的适用性、通用性和可用性。
	% \end{enumerate}

	% 本文目标系统的现已开发完成,并在企业中投入了应用。应用结果表明,该系统可以为各类面向 C/C++ 语言的静态分析工具提供更为精准有效的自动化评估测评功能。

	\iffalse
		本文针对企业在复杂代码场景中使用 C/C++ 静态分析工具时,
		普遍存在依赖人工构造测试场景、
		覆盖能力不足且评估结果难以量化的问题,
		提出了一种面向具体规则、
		基于缺陷场景的静态分析工具覆盖能力评估方法,
		并设计实现了相应的自动化评估系统。

		为突破传统评估方式以静态用例为核心、
		难以反映分析能力边界的局限,
		本文从静态分析工具的通用分析假设与能力约束出发,
		将工具覆盖能力建模为其在一组结构化缺陷场景空间中的可检测性问题。
		围绕该建模思路,
		本文设计了一种具备演化能力的场景知识库,
		并引入检索增强生成技术,
		以支持复杂代码场景的自动构造与持续扩展,
		从而实现对静态分析工具分析精度的动态评估。

		在系统实现层面,
		本文设计并实现了包含场景生成、
		场景管理与评估计算等功能模块的评估系统原型。
		该系统能够面向不同静态分析规则,
		自动构造跨文件、深调用链等具有代表性的复杂缺陷场景,
		并在此基础上对工具的精准度、
		召回率及 F1 指标进行量化计算。

		考虑到本文研究对象主要面向企业内部使用的多种 C/C++ 静态分析工具,
		本文选取开源工具 clang-tidy 作为实验对象,
		对所提出的评估方法和系统进行了可验证实验。
		实验结果表明,
		该方法能够有效刻画静态分析工具在不同规则维度下的覆盖能力特征,
		具备良好的适用性和通用性。

		本文的研究成果已在企业环境中得到实际应用,
		可为 C/C++ 静态分析工具的评估、
		选型及优化提供更加客观、可量化的决策依据。
	\fi
	todo
\end{abstract}

\begin{abstract*}
	todo
\end{abstract*}

\tableofcontents
% \listoffigures
% \listoftables
\listoffiguresandtables
% \include{chapters/notation}

% 三条铁律(请记住)

% Research 只写“不足”,不写“原理”
% EvaluageModel 只讲“抽象”,不讲“实现”
% System / Implementation 永远服务于“评估模型成立”


\mainmatter
% !TeX root = ../main.tex

\chapter{绪论}

% 多写就多写,以后再删

\section{研究背景与意义}

% 软件安全很重要,c/c++在软件中的占比大
随着软件在应用领域的广泛使用和迅猛发展,软件安全至关重要。上至国家安全,下至百姓生活,软件的缺陷都意味着或大或小的损失。在软件不断发展的过程中,减少软件缺陷是一个重大的课题。在软件语言家族中,C/C++占据比较大的比例,并且这类软件往往承担重要的低层功能,是构建操作系统、大型游戏引擎、数据库、嵌入式系统等对性能、效率和底层硬件控制有极致要求的软件的基石。C/C++软件的安全可信应该得到足够的重视。

% 静态分析是保障软件安全的重要一环
为了减少软件缺陷,静态分析技术被开发人员广泛使用。静态分析是指通过静态地检查代码本身而非运行程序,来分析软件的性质,静态分析可以应用在软件开发的各个阶段。静态分析技术依赖形式化的原理,可以实现快速分析,能够在分析精度和速度之间找到平衡。另外,由于软件缺陷发现的越晚,造成的成本越高,而静态分析技术能在软件开发多生命周期发挥作用,所以静态分析技术能够尽早地发现并帮助修复软件漏洞,以降低缺陷造成的开发成本和实质安全成本。由于速度快、不需要实际运行等优势,静态分析技术成为了C/C++软件的缺陷检测的常用工具,显著控制了软件缺陷。

% 提高静态分析质量和效率,能够帮助保障软件安全
在追求安全性和可靠性的大环境下,软件行业对清楚的了解静态分析工具的分析精度提出了更高的要求。高效充分评估静态分析工具的检测能力,是使用静态分析工具时的迫切需求,是保障软件质量的重要环节。只有深入了解工具的实际检测能力,才能在面对大规模代码的静态分析中,实现更加精准和高效的质量保障。

\subsection{C/C++静态分析工具及其应用现状}
% 有哪些工具? 应用现状如何?

学术界和工业界在C/C++静态分析领域已做了大量的研究,开发出了众多静态分析工具,在软件开发的过程中得到了广泛的应用。现有的静态分析工具众多,可以根据多个维度进行划分cite:

\begin{enumerate}
	\item \textbf{系统建模}。系统建模是静态分析工具对目标的抽象过程,即工具如何刻画目标程序,具体方法包括图结构、自动机结构、谓词抽象技术等等。
	\item \textbf{属性描述}。属性描述是静态分析工具如何建立不同的属性描述方法,即工具如何描述要检测的属性。具体来说,通常用有限自动机描述时序安全漏洞;用变量间的数值依赖关系和控制依赖关系精确刻画整数溢出、缓冲区溢出等。
	\item \textbf{检测过程}。检测过程指静态分析工具检测目标程序并得出分析结果的过程。常用的有基于词法模式匹配的方法、基于路径的方法、基于子图的方法、基于自动机的方法等。
	\item \textbf{结果验证}。结果验证指的是静态分析工具用何种方式去降低误报率、提高分析精度,通常有基于理论证明器的方法,基于SAT求解器的方法等。
\end{enumerate}

在不同的检查过程中,静态分析工具都有着不同的方法。众多工具遵循相同或不同的原理,针对软件缺陷检测提出他们自己的检查方法,比如下列几个工具在上述维度中的位置如表\ref{tab:position_of_some_tools}所示,其他工具总能在这四个维度上找到他们的位置。工具发展非常迅速,一次对现状的分类无法准确描述当前工具的多样。

\begin{table}
	\centering
	\bicaption
	{部分静态分析工具在四个维度中的位置}
	{Positions of Some Static Analysis Tools in Four Dimensions}
	\label{tab:position_of_some_tools}
	\begin{tabular}{ccccc}
		\hline
		工具         & 系统建模   & 属性描述   & 检测过程   & 结果验证   \\
		\hline
		Flawfinder & 字符串    & 字符串    & 模式匹配   & 无      \\
		UNO        & 自动机    & 自动机    & 自动机求积  & 无      \\
		Saturn     & 布尔程序   & 自动机    & 基于子图   & SAT    \\
		BLAST      & 谓词程序   & 自动机    & 基于路径   & SAT    \\
		\ldots     & \ldots & \ldots & \ldots & \ldots \\
		\hline
	\end{tabular}
\end{table}

尽管静态分析具备显著的优势,但其存在理论上的“不可能三角”:分析资源消耗、分析速度和分析精度三者不可同时达到最佳状态,只能在其中寻找平衡。这一限制决定了静态分析工具对代码问题的分析精度有不确定的表现,在面对企业级的庞大代码库时,表现出的能力往往是一个复杂的“黑盒”。对C/C++静态分析工具来说,特别是面对多文件交叉、函数调用层次较深或控制流程复杂的代码时,各种工具的检测能力差异尤为显著。

\subsection{静态分析工具评估的国内外研究现状}

% 评估是本论文的关注点,那么别人对评估的研究现状如何?

目前研究人员对静态分析技术的研究比较充足,但是对静态分析工具的评估研究相对较少。首先是对静态分析工具的评估局限于一些静态且单次的测试,这不能随着工具的更新和发展准确的表现工具能力。

\subsubsection{一次性评估结果}
有较多研究集中在对工具进行一次性评估。文献。。。。文献。。。

\subsubsection{评估框架}
目前评估框架有如下研究。。

\subsubsection{缺陷测试集}
测试集是用于评估静态分析工具的最直接的方法。包括。。。

\subsection{研究意义}
对静态分析工具进行评估,研究评估方法,设计出一套模块化评估系统据


\section{本文主要工作内容}
\section{本文工作的创新性及主要技术特色}
\section{论文组织与章节结构安排}

本文共分为六个章节。

第一章为绪论。本章首先介绍了针对C/C++语言的静态分析在软件行业中的应用和研究现状,简单阐述了对静态分析工具进行系统化评估的重要性,然后介绍了本文的主要工作和组织架构。

第二章为研究现状。本章对静态分析和代码生成的研究现状进行罗列和分析,分析他们的不足之处和可以进行结合的点。本章为本文的评估方法和系统设计作理论铺垫。

第三章为场景化覆盖能力评估问题建模。本章提出一种场景化的评估方法,研究了静态分析工具的原理和对应的能力边界,基于能力边界引出场景空间的结构化建模方法,通过形式化的建模过程阐述了场景化评估方法的合理性和系统性,为静态分析工具的评估系统设计提供理论基础,并分析了这种基于场景的评估方法的优势与局限。

第四章为面向场景的评估系统。本章基于前文对评估问题的建模,设计出一套模块化评估系统,基于一个场景数据库,设计两个模块分别进行生成和管理,并介绍了评估指标计算方法,最后介绍了评估系统在实现中的具体技术手段和模块化优势。

第五章为实验和分析。本章针对静态分析评估系统的目标需求,设计了针对性的实验,使用clang-tidy作为实验静态分析工具,将其规则进行细致评估和分析,呈现系统的效果和多工具可拓展性。最后基于实验结果C/C++静态分析工具评估问题进行了优势和不足分析。

% 绪论
%   研究背景
%       静态分析的应用
%       评估需求
%   研究意义
%   主要工作
%   组织架构
% !TeX root = ../main.tex

% Research
% → 证明现有方法和技术背景不足以解决你的问题

\chapter{论文相关技术}

本章在后文进入评估系统设计和实现之前介绍一些相关技术原理,旨在建立基本的技术共识。

\section{\tc~静态分析工具及其原理}
\label{sec:\tc~静态分析工具及其原理}
% 工具分类
学术界和工业界在\tc~静态分析领域已做了大量的研究,
开发出了众多静态分析工具,并且发展非常迅速。
如今商业工具有
LDRA Testbed、
Perforce Klocwork、
Parasoft \tc~test、
Synopsys Coverity等;
开源工具有
Flawfinder、
FindBugs、
SonarQube、
Clang-tidy等。
许多公司也在基于一些静态分析框架研究开发属于自己的静态分析工具。

对于如此众多的静态分析工具,
可以根据多个维度进行划分:

\begin{itemize}
	\item 系统建模。
	      系统建模是静态分析工具将真实程序映射为可分析抽象模型的过程,
	      其核心目标是在分析可行性与精度之间取得平衡,
	      不同工具在系统建模阶段对程序语义的抽象程度不同,
	      直接决定了后续分析能力的上限。
	      具体方法包括图结构、自动机、布尔程序、谓词抽象程序等等。
	\item 属性描述。
	      属性描述指静态分析工具对待检测程序性质的形式化表达方式,
	      即工具如何描述要检测的属性,
	      属性描述能力限制了工具能够检测的缺陷“类型空间”。
	      具体来说,通常用有限自动机描述时序安全缺陷、
	      用变量间的数值依赖关系和控制依赖关系精确刻画整数溢出、缓冲区溢出等。
	\item 检测过程。
	      检测过程指静态分析工具在抽象模型上执行分析、
	      推导程序行为并发现潜在缺陷的过程。
	      检测过程的选择直接影响分析的复杂度、可扩展性及误报率。
	      常用的有规则匹配、数据流分析、路径枚举与剪枝、
	      子图同构匹配、自动机求积和符号执行等。
	\item 结果验证。
	      结果验证用于判断分析结果是否为真实缺陷,
	      并通过约束求解或反例分析降低误报,
	      通常与路径敏感分析相结合,
	      在规则驱动型工具中较少使用。
	      通常有基于理论证明器的方法,
	      基于SAT(Satisfiability)求解器的方法等。
\end{itemize}

在不同的检查过程中,静态分析工具都有着不同的方法。
众多工具遵循相同或不同的原理,针对软件缺陷检测提出他们自己的检查方法,
比如下列几个工具在上述维度中的位置如表\ref{tab:position_of_some_tools}所示,
其他工具总能在这四个维度上找到他们的位置。
工具发展非常迅速,一次对现状的分类无法准确描述当前工具的多样。

\begin{table}
	\centering
	\caption
	{部分静态分析工具在四个维度中的位置}
	\label{tab:position_of_some_tools}
	\begin{tabular}{ccccc}
		\hline
		工具         & 系统建模         & 属性描述       & 检测过程     & 结果验证   \\
		\hline
		Flawfinder & 字符串          & 字符串        & 规则匹配     & 无      \\
		% UNO        & 自动机          & 自动机        & 自动机求积    & 无      \\
		Clang-tidy & AST + 局部 CFG & AST 模式语义约束 & 规则匹配     & 无      \\
		CSA        & CFG          & 隐式状态断言     & 路径敏感符号执行 & 理论证明器  \\
		% Saturn     & 布尔程序         & 自动机        & 基于子图     & SAT求解器 \\
		% BLAST      & 谓词抽象程序       & 自动机        & 基于路径     & SAT求解器 \\
		\ldots     & \ldots       & \ldots     & \ldots   & \ldots \\
		\hline
	\end{tabular}
\end{table}

% CFG
控制流图(Control Flow Graph, CFG)指的是一个图结构,
其中节点代表基本块(Basic Block),
基本块指的是一系列连续执行的指令,
中间不会有边,
边代表了指令从一个基本块到另一个基本块的可能跳转。
CFG理论上可以覆盖所有的程序执行路径,
但由于程序结构往往非常复杂,
包含众多分支、循环和函数调用,
分析路径呈指数级别增长,
这就是路径爆炸(Path Explosion)。

% 路径敏感
路径敏感(Path-sensitive)是指工具在检测代码问题时,
能够追踪并区分程序中由条件分支(如 if/else)产生的不同执行路径及其对应的程序状态,从而更准确地判断漏洞或错误,避免了非路径敏感分析容易产生的误报(false positives)和漏报(false negatives),但计算复杂度更高。

% 数据流
数据流分析(Data Flow Analysis)是基于CFG的分析方法,
它定义传递函数来描述状态转移规则,
描述在某个基本块内,
数据流信息如何被单条指令更新。
沿着CFG传播状态信息迭代求解直到收敛,
核心在于追踪程序中数据状态在不同执行路径上的流动和变化,
最终检测代码缺陷或进行程序理解。

% 符号执行
符号执行(Symbolic Execution)技术也基于CFG,
它可以通过分析技术得到让特定区域执行的输入。
它通过使用抽象的符号代替具体值来模拟程序的执行,
当遇到分支语句时,
它会探索每一个分支,
将分支条件加入到相应的路径约束中,
若约束可解,
则说明该路径是可达的。
符号执行的目的是在给定的时间内,尽可能的探索更多的路径。

% AST
AST(Abstract Syntax Tree,抽象语法树) 是静态程序分析的核心数据结构。
它以树状形式表示源代码的语法结构,
由编译器前端(如 Clang、Csc、Babel)将源代码解析产生,
AST 会省去无用的符号,仅保留代码的语义结构。
AST中节点代表代码中的元素,
如 VariableDeclaration(变量声明)、
BinaryExpression(二元表达式)、
IfStatement(If 语句)等,
边代表元素之间的层级关系,
如函数包含语句,语句包含表达式等。

% 以规则为单元
软件行业在实际使用静态分析工具进行检查时,往往以工具规则为单位进行检查,
规则也是直接面向程序的静态分析工具设计时的基本单元。
每个工具都有擅长的缺陷检测领域,它会为此设计多种规则,
每个规则都对应这个领域的某一部分缺陷。
例如开源工具Clang-tidy主要面向的是代码违规、
接口误用等可以通过静态分析推断出的缺陷,
它的规则包括以bugprone-开头的针对易出错代码结构的检查、
以concurrency-开头的与并发编程相关的检查(包括线程、纤维、协程等)等。

% 广义的静态分析的不足
尽管静态分析具备显著的优势,
但也有很多广义上的不足。
首先,静态分析存在理论上的“不可能三角”:
分析资源消耗、分析速度和分析精度三者不可同时达到最佳状态。
基于精准的CFG和数据流分析,
在理论上可以实现精准分析,
但由于路径爆炸的存在,
静态分析工具往往会牺牲一些分析精度,
在“不可能三角”中找到工程化平衡点。
这一限制决定了静态分析工具对代码问题的分析精度有不确定的表现,
在面对企业级的庞大代码库时,
表现出的能力往往是一个复杂的“黑盒”。
其次,静态分析技术在实际应用中也会产生诸如误报率高、
精确度较低、告警合并和处理比较复杂等问题。
最后,为了更好的保障安全,大规模软件在进行静态分析检测的时候,
往往会结合多个工具,
在工具本身的检测精度之外,
还有告警处理、数据整合等问题会影响实际应用效果。

% c/cpp对静态分析工具的分析带来的不足
对\tc~静态分析工具来说,也存在基于语言特性带来的不足。
第一,\tc~语言因其底层操作能力强、资源控制精细、广泛用于企业级大规模软件开发,
因此一次完整的静态分析可能长达若干小时。
第二,由于各种工具的检察原理不同,面对多文件交叉、函数调用层次较深或控制流程复杂的代码时,
各种静态分析工具的检测能力差异尤为显著。
第三,\tc~静态分析工具需要面对内存越界访问、资源泄漏、
空指针访问、\tc~代码规范、以及定制化检查规则,
这些特殊的缺陷往往有特化的触发场景。
特别是面对多文件交叉、函数调用层次较深或控制流程复杂的缺陷场景时,
各种工具的检测能力差异尤为显著。

具体来说,各种工具在面对如下\tc~问题会有明显的检查瓶颈:
\begin{itemize}
	\item 跨文件。
	      跨文件的程序会大幅增加代码的规模,
	      涉及到的全局变量、函数和类的数量众多,这使得分析复杂度呈指数级增长。
	      跨文件调用的上下文可能丢失,如函数的定义和调用在不同文件中,
	      参数的精确值和调用栈的信息很难还原。

	\item 函数调用层数深。
	      调用深层次的函数时,涉及多层函数间的调用关系,增加了解析每层语义的难度。
	      另外,静态分析工具对函数调用通常采用上下文敏感分析,
	      即根据具体的调用点来分析函数行为。
	      如果调用层次过深,则需要追踪调用链中的所有上下文信息,
	      分析过程会更加复杂且容易失去精度。
	      另外在深层调用中,递归分析的复杂度急剧增加,可能无法准确推导出函数最终的影响。

	\item 程序基本块太多。
	      程序基本块是一段线性的程序码,只能从这段程式码开始处进入这段程序。
	      基本块太多会导致路径组合爆炸,在静态分析中,基本块多意味着可能的路径数急剧增加。
	      如果程序中有许多分支和循环,每条路径都需要单独分析,导致路径数量指数增长,
	      进而导致时间和空间开销过大,在不可能三角的平衡下丢失精度。

	\item 指针断链。
	      指针断链指的指针原本指向一个有效的内存地址,
	      但后来所指的内存被释放或失效后,指针依然保持原来的值并试图被访问。
	      分析工具需要在跨函数、跨模块时保持准确的指针追踪,
	      单独分析某一个部分通常无法捕获完整的指针使用语境,
	      导致程序分析工具在面对指针断链问题时展现出不足。

	\item 代码重入次数(max loop num,MLN)太大。
	      大多数静态分析工具在对循环进行检查时,往往使用展开方式,
	      将循环展开 MLN 次,以检查循环代码在重入多次时的安全问题,
	      但该参数对于检查性能来说影响较大,静态分析工具一般将它设置为较小值,
	      对于确保重入代码的安全性来说保障性有待检验。

\end{itemize}

本节介绍了静态分析工具及其基本原理。
总的来说,静态分析工具种类繁多,
但目前静态分析工具仍然有有广阔的优化空间,
为此,对静态分析工具的评估和相关的研究不仅对工具优化至关重要,
也对软件安全有着长远的影响。

\section{静态分析工具评估}

一般来说,对静态分析工具的评估大都使用测试集实测的方式,评估方法如下。

\begin{enumerate}
	\item 明确评估对象与测试集。
	      测试集指的是对工具进行评估时准备的缺陷或测试用例全集。
	      定义被评估的静态分析工具和测试集,设定衡量工具能力的评估指标。
	      例如,测试哪些具体测例文件,并通过哪些标准衡量性能。
	      格式化存储基准测试集中的关键信息,例如文件数量、缺陷类型、注入缺陷位置等,
	      为后续评估过程提供数据支持。

	\item 运行静态分析工具并获取原始结果。
	      利用自动化工具高效批量执行测试,采集分析结果,
	      提高运行效率的同时减少手动操作带来的误差。

	\item 解析结果并计算评估指标。
	      对原始结果中的告警信息进行处理与分类,计算并对比各类评估指标,
	      对工具能力作出全面分析。同时在这一步需要对缺陷具体代码进行归档。
\end{enumerate}

\subsection{测试集}
\label{sec:测试集}

测试集是用于评估静态分析工具的基础内容。
% \textbf{场景}(Scene)描述了缺陷发生的实际条件和路径,一种缺陷可以关联多个场景。
测试集的完整性和复杂性在静态分析工具能力评估中尤为关键,它需要能够体现实际软件的复杂场景。
若测试集包含的缺陷场景有限,
评估结果可能无法反映工具在真实环境中的表现,
导致实际应用中的缺陷频发,
评估系统应该避免这种情况。

常见缺陷枚举(Common Weakness Enumeration,CWE)
是开源社区维护的常见软件与硬件弱点列表,
它并非仅针对静态分析工具,而是涵盖广义的软件缺陷集合。
CWE 共包含 944 种计算机软硬件相关缺陷,
其中软件领域缺陷 399 种,与\tc~相关的缺陷共 118 种。
对于所有缺陷,CWE 从分类层级、基础类型、变量、作用链等多个维度进行组织;
但针对\tc~相关的 CWE 缺陷,本文将这 118 种缺陷视为并列的缺陷集合进行分析。
每个 CWE 缺陷均具有唯一编号,常见的\tc~缺陷包括
\texttt{CWE-126:Buffer\_OverRead}(越界读),
\texttt{CWE-416:Use\_After\_Free}(释放后使用)等。

CWE缺陷仅仅是缺陷分类,
并不能精准的识别到细致的场景,评估粒度不够细。
一个具体的评估测试集选择应该以CWE缺陷为基础,建立场景粒度的测试集。

朱丽叶测试集(Juliet Test Suite,JTS)
是由美国国家安全局(NSA)软件质量中心(CAS)为评估静态分析工具的能力而创建的专用测试集,
它基于CWE缺陷分类,也有自身细化的代码分类。
JTS 由 61387 个测试用例组成,大部分测试用例使用定制测试用例模板引擎生成,
其他部分的测试用例由人工创建而成。
该测试集涵盖了 118 种CWE类型,由 96896 个\tc~文件、867 万行代码构成。

JTS 在行业中取得了一定的成功,
定下了较好的基准,
在很多研究人员的静态分析标准评估流程中都有使用,
但经过研究,JTS也存在如下局限性。

\begin{itemize}
	\item JTS 测试集主要集中于单个文件内的代码缺陷,
	      对于跨文件和跨编译单元的代码缺陷的用例很少,
	      而这种情况恰恰是大型企业级软件的缺陷发生的多数情况。
	\item JTS 测试集是存量代码,
	      对于大型企业级软件中实际发生的问题有其更新机制,
	      但流程较长,更新不及时,
	      且依赖非常多的人力投入。
	\item JTS 测试集并不直接反馈测试人员对某种特定场景的需求,
	      当测试人员需要提供“某工具能否覆盖某种场景的说明和检验”时,
	      需要花较多成本检索或构建。
	      % \item JTS 测试集不能反映研究人员对特定静态分析工具的测试需求,
	      %       某些产品代码中有特殊的定义、规范和检查要求,
	      %       JTS 不能直接反映这些问题的检查精度。
\end{itemize}

Functional类型是JTS在同种CWE缺陷的基础上对缺陷类型的进一步细分。
每个CWE缺陷的 Functional 类型不同,
如果该 CWE 类型没有更细粒度的内容,则 Functional 类型为 basic。
测试用例中使用的数据结构、特殊函数或特殊标记都有可能是 Functional 类型中的关键词。
举例来说,
\texttt{CWE-667:Improper\_Lockin} (加锁不当)这个缺陷,
仅有 basic 作为 Functional 类型;
\texttt{CWE-126:Buffer\_OverRead} (越界读内存)这个缺陷,
包括
\texttt{char\_alloca\_loop}、
\texttt{char\_alloca\_memcpy}、
\texttt{char\_alloca\_memmove}
等多种 Functional 类型。

代码结构复杂度类型(Flow)是JTS设计的代码结构中的数据流类型、控制流类型。
这是另一种对缺陷的分类维度,
更针对静态分析的工具原理,能够对工具的能力进行更细致的划分。
Flow 类型共有48种,用数字标记,
01代表 baseline ,02至30代表控制流,31至84代表数据流,
其中部分列举如表\ref{tab:Meaning_of_some_Flow_types}
JTS使用Functional类型和Flow类型作为文件名,
用以组织测试用例。

\begin{table}
	\centering
	\caption
	{部分Flow类型含义}
	\label{tab:Meaning_of_some_Flow_types}
	\begin{tabular}{ccp{9cm}}
		\hline
		Flow 类型
		 & 所属类别
		 & 该 Flow 类型代码的主要结构特点                                                                                                           \\
		\hline
		02
		 & 控制流
		 & \texttt{if(1) and if(0) }                                                                                                    \\
		08
		 & 控制流
		 & \texttt{if(staticReturnsTrue()) and if(staticReturnsFalse()) }                                                               \\
		14
		 & 控制流
		 & \texttt{if(globalFive==5) and if(globalFive!=5) }                                                                            \\
		18
		 & 控制流
		 & \texttt{goto statements}                                                                                                     \\
		31
		 & 数据流
		 & \texttt{Data flow using a copy of data within the same function}                                                             \\
		33
		 & 数据流
		 & \texttt{Use of a C++ reference to data within the same function }                                                            \\
		65
		 & 控制流、数据流
		 & \texttt{Data passed as an argument from one function to a function in a different source file called via a function pointer} \\
		\ldots
		 & \ldots
		 & \ldots                                                                                                                       \\
		\hline
	\end{tabular}
\end{table}

\subsection{评估指标}
\label{sec:评估指标}

静态分析工具的性能分为很多维度,用对应的指标来表征。
在执行完检查之后,指标通过“有无告警”和“有无缺陷”矩阵数据
表\ref{tab:statistics_of_analysis}来计算。

\begin{table}
	\centering
	\caption
	{静态分析检查得到的数据}
	\label{tab:statistics_of_analysis}
	\begin{tabular}{p{2cm}p{5cm}p{5cm}}
		\hline
		    & 有告警                   & 无告警                   \\
		\hline
		有缺陷 & TP(True Positive,正报)  & FN(False Nagetive,漏报) \\
		无缺陷 & FP(False Positive,误报) & TN(True Nagetive)     \\
		\hline
	\end{tabular}
\end{table}

精确度(Precision)指工具所有告警中实际为缺陷的比例,
其计算方法如公式(\ref{eq:Precision})所示。
精确度回答了“静态分析工具的缺陷告警是否精确”的问题,描述了静态工具的可信程度。
精确度越高,代表静态分析工具给出的告警更有可能确实是缺陷,即该告警的可信度越高。

\begin{equation}
	\text{Precision} = \frac{TP}{TP + FP}
	\label{eq:Precision}
\end{equation}

召回率(Recall)指工具正确告警缺陷占所有缺陷的比例,
其计算方法如公式(\ref{eq:Recall})所示。
召回率回答了“静态分析工具能识别多少缺陷”的问题,
描述了工具告警对于整体缺陷的覆盖程度,反映工具的漏报情况。
召回率越高,代表静态分析工具找到了较多的缺陷,漏报率低。

\begin{equation}
	\text{Recall} = \frac{TP}{TP + FN}
	\label{eq:Recall}
\end{equation}

F1分数(F1)是精确度和召回率的调和平均,为综合指标,
其计算方法如公式(\ref{eq:F1})所示。
F1 分数反映静态分析工具在漏报和误报之间的平衡性。
一个工具如果采用激进策略,选择将所有疑似缺陷都进行告警,
则该工具尽管拥有非常高的召回率,但其精准度会非常低,
在这种情况下,F1 分数可以更好地反映工具的缺陷分析能力。

\begin{equation}
	\text{F1} = \frac{2 * Precision * Recall}{Precision + Recall}
	\label{eq:F1}
\end{equation}

区分度(Discrimination Rate)
被用来评估静态分析工具辨别“有缺陷程序结构”和“无缺陷程序结构”的能力,
其计算方法如公式(\ref{eq:Discrimination})所示。
NSA CAS(美国国家安全局 软件质量中心)引入了“有区分度的正报”这个概念,
即Discriminations,其含义是工具正确地报出某测试用例的“有缺陷函数”,
同时没有错误地报出该测试用例的“无缺陷函数”。
对于这种告警,开发人员再修改正确之后告警会自动消失。

\begin{equation}
	\text{Discrimination} = \frac{Discriminations}{TP + FN}
	\label{eq:Discrimination}
\end{equation}

CWE覆盖度(CWE Coverage)
是工具能够检测的缺陷类数和总缺陷类数的比值,
其计算方法如公式(\ref{eq:CWE_Coverage})所示。
用来评估静态分析工具在缺陷粒度下的覆盖能力。

\begin{equation}
	\text{CWE\_Coverage}
	= \frac{\text{CWE\_Covered}}{\text{CWE\_total}}
	\label{eq:CWE_Coverage}
\end{equation}


Flow覆盖度(Flow Coverage)
是指工具的某一个规则在它所负责的CWE范围中能够检测出的Flow数和总Flow数的比值,
其计算方法如公式(\ref{eq:Flow_Coverage})所示。
用来评估静态分析工具的每个规则在它负责的缺陷下对具体出错类型的覆盖能力。

\begin{equation}
	\text{Flow\_Coverage}
	= \frac{\text{Flow\_Covered}}{\text{Flow\_total}}
	\label{eq:Flow_Coverage}
\end{equation}

% 指标总结todo

其他评估指标还有检查速度、运行时算力消耗、检查功能等等,
与静态分析工具的覆盖能力不直接相关。
本文研究聚焦于评估静态分析工具正确检测软件缺陷的能力,
也就是上述工具的核心评估指标。

本节介绍了用于静态分析工具评估的流程、数据集和指标,
为后文系统化评估提供了基本知识共识,
也启发了更完善的评估方式。

\section{测试集拓展和生成式人工智能}
\label{sec:测试集拓展和生成式人工智能}

用例集的选择、使用及不断扩充,对于静态代码分析工具评估非常重要。
本文在章节\ref{sec:静态分析工具评估的国内外研究现状}部分提到测试集拓展的研究,
他们的研究的优点在于可以根据某些预设的场景进行拓展,
但缺点在于有一些静态分析工具的检查边界依然没有被触及到。
另一方面,这些场景测例没有被系统化的管理起来,
所以针对静态分析工具评估的测例拓展和管理,
需要使用更先进的技术。

随着人工智能(Artificial Intelligence, AI)技术的发展,
未来基于生成式技术的场景生成方法有望帮助测试集的构建,
尤其是检索增强生成技术(Retrieval-augmented generation,RAG),
被广泛应用于问答和生成任务中,
特别是在大规模参考文档场景下,RAG能够有效地整合分散的知识源,
自动生成高质量的内容,
可以启发测试集拓展,
无论是正向梳理,还是被动补充,
都有RAG的用武之地。

LLM 是由具有大量参数的人工神经网络组成的一类语言模型。
目前市面上有海量对LLM的研究,
各种LLM的能力不同,举例来说,
闭源的GPT4o有超强的多模态功能,
可以细腻的感知情绪并回答;
而在代码生成方面,
有LLama3系列、DeekSeek v2系列、Qwen2系列都广为使用。
从目前各家开放的技术报告来看,
对于代码能力谈的都比较宽泛,
涉及细节较少。

RAG 基于大语言模型(large language model,LLM),
RAG 依赖于与用户输入相关的背景知识,
在 LLM 和检索能力的基础上构建索引、检索和生成过程。
索引过程是构建背景知识向量数据库的过程,
它将不同格式的数据统一化,
随后分块并使用嵌入模型储存入向量数据库。
检索过程则是从数据库中提取相关知识的过程,
该过程利用嵌入模型将用户询问向量化,
随后与向量数据库比较相似度,
检索出达到相似度阈值的数据。
检索过程是提升 RAG 表现的关键步骤,
由于 LLM 的知识有限,
从已有数据库中检索出问题所需的知识非常关键,
常用的检索方法包括 BM25 和向量检索,
分别对应精确搜索和自然语言理解搜索。

其中BM25是一种经典的信息检索算法,
用于评估用户查询与文档之间的相关性,
广泛应用于搜索引擎(如Elasticsearch)中,
它改进了传统的TF-IDF,
通过综合考虑词频(TF)、逆文档频率(IDF)以及文档长度等因素来打分,
核心思想是文档包含查询词越多、越罕见(IDF高)且文档不是特别长时,
相关性得分越高。
BM25检索速度不快,
但是可以找到关键词非常相关的内容。
另一方面,
向量检索是一种基于语义相似度而非传统关键词匹配的搜索技术。
通过深度学习模型将非结构化数据转化为一串数字数组(高维向量),
这些数字捕捉了数据的内在语义特征。
在检索时将查询请求也转化为向量,
在多维向量空间中计算查询向量与已知数据向量之间的“距离”,
常用距离包括余弦相似度、欧氏距离和内积等,
向量检索速度快,
可以快速找到语义相似的语句。

生成过程则将检索到的数据与用户输入整合,
形成最终的提示词,
提供大语言模型以生成输出。

典型的复合了向量检索和BM25检索的RAG流程如图\ref{fig:复合检索RAG}所示。

\begin{figure}[ht]
	\centering
	\includegraphics[width=0.9\textwidth]{复合检索RAG.png}
	\caption{复合检索RAG}
	\label{fig:复合检索RAG}
\end{figure}

RAG 的效果相比纯粹使用 LLM 有所提升,
但是对于用于静态分析工具评估测试集来说,
仍然需要进一步验证和确保结果的正确性。
为此 RAG 可以和责任链模式(Chain of Responsibility,COR)一起使用
其原理如图\ref{fig:责任链}所示。
COR 是一种通用行为设计模式,
允许请求沿着处理者链进行发送,
收到请求后,
每个处理者均可对请求进行处理,
或将其传递给链上的下个处理者。
COR 常见的应用方式是创建一系列定制的校验方法,
让某个对象一次经过这些校验方法,
以检验对象的符合要求的程度。

\begin{figure}[ht]
	\centering
	\includegraphics[width=0.9\textwidth]{责任链.png}
	\caption{基本的责任链模式}
	\label{fig:责任链}
\end{figure}

本小节从测试集拓展的目的出发,
介绍了生成式人工智能的基本原理以及有何优势,
RAG可以看作当前应用人工智能领域用于生成的法宝,
结合COR等用于校验的软件设计模式,
可以为静态分析评估测试集拓展所用。

\section{模块化和容器化}

经过前文的介绍,
静态分析工具自动化评估系统需要的技术包含多个方面,
实际运行评估、管理测试集、拓展测试集、展示测试结果
各自都有独立的实现环境和方法,
并且需要建立在多样的生产环境中,
为此,模块化设计方法和容器化技术不可或缺。

模块化设计,旨在于将一个系统细分为许多小单元,
称为模组(module)或模块(block),
可以独立的于不同的系统中建立与使用。
模块化设计的特征为将功能切分为抽象的、可扩充的、可重复使用的模块;
模块需严格使用定义明确的模块化接口,
并以行业间标准作为接口规范。
模块化设计通常使用主程序、子程序和子过程等架构
将软体的主要结构和流程描述出来,
而非一般程式的编写──逐条输入电脑程式和指令。

容器化是一种软件打包和部署技术,
它将应用程序代码及其所有依赖项封装到一个标准化的、
可移植的“容器”中,
使应用能在任何环境(开发、测试、生产)中一致地运行。
其中容器(Container)指的是包含应用及其运行环境的独立软件包,
容器镜像 (Image)指的是容器的蓝图或模板,包含了创建容器所需的一切文件,
容器引擎 (Engine)指如Docker等用于创建、运行和管理容器的平台。

容器可以共享主机的操作系统内核,比虚拟机更轻量、高效,
实现“一次构建,随处运行”,
它极大简化了应用开发、测试和部署,
非常适合静态分析评估这样需要多种功能集成,
且功能之间比较灵活的系统。

本节介绍了模块化设计和容器化技术,
为本文目标系统的设计和实现提供了基本的技术路线。

总结本章,
从\tc~静态分析工具的分类、原理,
到静态分析工具的评估指标和流程,
到一些本文系统可以用得到的技术,
本章提供了一个基本的技术共识,
本文后续的问题建模、
系统设计和实现都基于这些技术成果。
% 国内外研究现状
%   静态分析工具现有评估方式和问题
%   现有代码生成方法和问题
% !TeX root = ../main.tex


% EvaluateModel
% → 定义问题本身,给出评估的数学/结构化对象

\chapter{场景化覆盖能力评估问题建模}

本章针对静态分析工具的场景化覆盖能力评估进行抽象化讨论,
旨在明确基于场景空间的静态分析工具能力评估方法应该如何设计。

从前文可以得出,静态分析工具的评估重点在于测试集的设计和管理,
有了组织清晰的测试集,
实际测试的结果才可以清晰的展示静态分析工具的能力地图。
那么应该怎样设计测试集空间呢?
JTS给出了一些参考,
它使用Functional维度和FLow维度组织测试集,
这两个正交维度在每一个缺陷下都各自展开,
但是这样的维度仅能触及工具的一部分能力,
不能对其能力进行更细粒度的检查。

静态分析工具的检测能力不仅与自身原理和具体规则相关,
还与程序场景的结构特征高度相关。
为了对评估问题进行建模,
有必要从工具原理出发,
结合场景的代码结构,
抽象其在不同场景下的能力边界。
因此,本文对静态分析工具的检查能力进行细致分析,
并基于此设计场景空间维度。

本文的分析目标不以工具为单位,
而以工具的每一个规则为单位。
理由是工具不是单一能力,
每个规则都对应着不同的分析策略和能力边界,
要想对工具进行完整的评估,
必须以规则为粒度,以分析范围为限制,以能力边界为场景维度。

\section{静态分析工具的能力边界}

本文把静态分析工具的能力分为两类,
一类是工具所能分析的范围,
具体来说,包括静态分析工具规则所对应的缺陷范围,
以及规则所能检查的工程作用域。

第二类是基于原理的能力范围,
分为语义建模深度、
路径敏感程度、
路径状态空间控制程度、
约束精细化能力。

\subsection{分析缺陷范围边界}

每一个工具规则都对应着若干缺陷。
我们能够将每一个场景测例划分到唯一的缺陷中,
但不能把每一个规则划分到唯一的缺陷中,
规则对应的是一个缺陷子集,
该子集中的所有的测例都应该作为规则评估的测试对象。

例如Clang-tidy有一条规则,
名为

\iffalse
	在固定缺陷类型(或缺陷族)的前提下,评估工具在不同能力边界维度上的表现。
	缺陷类型决定了分析所需的最低语义能力,因此本文将缺陷类型作为评估前提,而非能力维度本身。
	场景要诱导静态分析工具错误理解
\fi

\begin{enumerate}
	从工具的检查能力出发,将场景空间分为以下维度,
	\item 缺陷:检查的是哪个缺陷 :约束条件/实验前提
	      不同缺陷对分析能力的要求完全不同:
	      空指针 → 路径敏感
	      缓冲区溢出 → 数值域 + 路径
	      API 误用 → 时序/状态机

	\item 分析作用域:在什么样的工程维度内分析?
	      刻度:
	      单函数
	      跨函数
	      上下文敏感跨函数
	      跨文件
	      全程序

	\item 语义建模深度:能理解到什么语义层、是否建模程序状态
	      刻度:无语义 → 语法级 → 控制流级 → 状态级

	      | 能力刻度 | 本质     |
	      | ---- | ------ |
	      | 无语义  | 文本匹配   |
	      | 语法级  | 只看结构   |
	      | 控制流级 | 理解执行顺序 |
	      | 状态级  | 跟踪变量取值 |

	\item 路径敏感程度:能区分多少执行路径?
	      刻度:
	      路径不敏感
	      分支敏感
	      循环敏感
	      局部路径敏感
	      完整的路径敏感

	\item 路径状态空间控制:如何抑制路径爆炸?能不能走得动复杂路径?
	      刻度:
	      无抑制
	      主动合并状态
	      限制路径深度
	      启用widen
	      {
	      widen:
	      当发现状态在“单调变化”时,
	      直接跳到一个“稳定上界”。

	      int i = 0;
	      while (i < 1000) {
			      i++;
		      }
	      use(i);

	      无 widen:路径/状态爆炸
	      有 widen:i >= 0,分析快速收敛
	      }
	      有非平凡循环处理能力
	      {
	      max_loop_num 属于 工程控制参数
	      “是否能处理非平凡循环” 属于 能力边界判断
	      }


	\item 约束精细化能力:是否验证路径可达性,仅做是否判断
	      刻度:
	      不约束
	      约束
\end{enumerate}

\section{场景空间的结构化建模}

为了系统化刻画影响静态分析工具检测能力的因素,本文引入场景空间的概念……

\section{场景空间驱动的评估方法的优势与局限}
场景驱动评估并非替代传统测试集方法,而是提供了一种从能力视角审视评估问题的补充路径。

% \subsection{二级节标题}

% \subsubsection{三级节标题}

% \section{脚注}

% Lorem ipsum dolor sit amet, consectetur adipiscing elit, sed do eiusmod tempor
% incididunt ut labore et dolore magna aliqua.
% \footnote{Ut enim ad minim veniam, quis nostrud exercitation ullamco laboris
% 	nisi ut aliquip ex ea commodo consequat.
% 	Duis aute irure dolor in reprehenderit in voluptate velit esse cillum dolore
% 	eu fugiat nulla pariatur.}

% 场景化覆盖能力评估问题建模(只讲抽象)
%   静态分析工具的原理和对应的能力边界
%   场景空间的结构化建模(正交)
%   场景驱动评估方法的优势与局限
% !TeX root = ../main.tex

\chapter{面向场景的评估系统设计}

\section{系统总体架构与评估流程}
\section{场景工厂的设计思想与生成约束}
\section{静态分析工具评估}
\section{场景资产管理与演化机制}
\section{评估指标计算与结果解释}

% \subsection{二级节标题}

% \subsubsection{三级节标题}

% \section{脚注}

% Lorem ipsum dolor sit amet, consectetur adipiscing elit, sed do eiusmod tempor
% incididunt ut labore et dolore magna aliqua.
% \footnote{Ut enim ad minim veniam, quis nostrud exercitation ullamco laboris
% 	nisi ut aliquip ex ea commodo consequat.
% 	Duis aute irure dolor in reprehenderit in voluptate velit esse cillum dolore
% 	eu fugiat nulla pariatur.}

% 面向场景的评估系统设计(只讲框架)
%   系统总体架构与评估流程
%   场景工厂的设计思想与生成约束
%   场景资产管理与演化机制
%   评估指标计算与结果解释
% !TeX root = ../main.tex
\chapter{面向场景的评估系统实现}

在前文完成系统总体架构设计与核心模块功能分析的基础上,
本章将介绍面向场景的静态分析工具评估系统的具体实现过程。

本章基于场景知识库的设计,
围绕“场景生成—场景管理—评估计算”这一核心工程流程,
详细阐述系统关键功能模块的实现方法与工程细节,
首先介绍场景知识库的工程实现,
然后重点说明在引入大语言模型后,
系统如何在不确定性条件下尽量保证结果的质量;
随后说明场景资产在系统中的管理、去重与分类实现;
接着重点描述评估流程的自动化实现方式;
最后从工程角度分析系统在工具替换与功能扩展方面的实现特性。

\iffalse
	\chapter{面向场景的评估系统实现}

	\section{场景知识库与数据模型设计}
	\subsection{场景实体与核心字段设计}
	\subsection{场景空间维度的数据承载方式}
	\subsection{自然语言描述与结构化信息的映射}

	\section{场景生成机制实现}
	\subsection{基于自然语言的问题抽象}
	\subsection{正向梳理的场景生成流程}
	\subsection{被动补充的场景生成流程}
	\subsection{大语言模型不确定性的工程约束}

	\section{场景去重、分类与管理实现}
	\subsection{场景等价性与去重策略}
	\subsection{基于场景空间的分类实现}
	\subsection{场景生命周期管理}

	\section{评估流程自动化实现}
	\subsection{评估流程总体调度机制}
	\subsection{工具执行与结果采集}
	\subsection{覆盖度指标计算与结果归一化}

	\section{系统可扩展性与替换性分析}
	\subsection{分析工具的替换机制}
	\subsection{场景维度的扩展能力}
	\subsection{评估流程的模块解耦}
\fi

\section{场景知识库的实现}

场景知识库需要支持长期演化,
同时服务于上层三个模块的功能,
需要同时作为RAG的检索库和测试集库。
本节首先介绍场景实体的核心字段,
再介绍场景空间的信息如何由字段数据承载,
最后介绍如何通过自然语言将结构化信息和代码结合到一起。

\subsection{场景等实体的核心字段}

根据本文设计的ER图\ref{fig:ER图},
场景实体是整个知识库的核心单元,
一个场景实体需要记录代码、
空间信息、
属于哪个缺陷等信息,
表\ref{tab:场景实体核心字段}逐一介绍了场景实体的核心字段。

\begin{table}
	\centering
	\caption{场景实体核心字段}
	\label{tab:场景实体核心字段}
	\begin{tabular}{ccp{9cm}}
		\hline
		字段                   & 格式     & 含义                       \\
		\hline
		description          & string & 代码内容的自然语言描述              \\
		good                 & dict   & good代码内容,键值为file:content \\
		bad                  & dict   & bad代码内容,键值为file:content  \\
		weakness             & int    & 所属的缺陷CWE ID              \\
		Program\_Span        & int    & 场景空间的 $P$ 维度:程序跨度        \\
		Semantic\_Complexity & int    & 场景空间的 $S$ 维度:语义建模复杂度     \\
		Path\_Complexity     & int    & 场景空间的 $C$ 维度:路径结构复杂度     \\
		Path\_Depth          & int    & 场景空间的 $L$ 维度:路径深度与状态空间控制 \\
		Reachability         & int    & 场景空间的 $R$ 维度:有无不可达路径     \\
		\hline
	\end{tabular}
\end{table}

其中description、good、bad三者是场景基础信息,
其他的是场景空间信息。

在知识库中除了场景实体作为核心表,
还有覆盖表和应该覆盖表,
都是一个多对多的关系,
分别记录运行之后规则和场景的覆盖结果、
规则和缺陷之间应该覆盖的真实值。

为了满足系统大量的检索需求,
场景知识库使用ElasticSearch(ES)承载,
ES的检索功能包括向量检索和其他多种混合检索,
非常适合本文系统的需求。

\subsection{场景空间信息的数据承载方式}

场景在空间中的信息并不是连续的,
如章节\ref{sec:场景空间}所述,
也并不是每个场景都有所有的维度,
维度是根据缺陷语义而不同的。
根据本文所涉及的刻度,
场景利用不同字段中的枚举或数字来维护维度信息。
数字使用前缀编码的方式,
0代表最简单的情况,
也可以代表该维度是平凡的;
第一个非零数字表示刻度,
后面的数字表示刻度内部的子维度,
维度信息刻画如表\ref{tab:场景空间信息字段含义},
这些字段体现的是本文目前的刻度设计,
前缀编码的方式也适合后续拓展。

\begin{table}
	\centering
	\caption{场景空间信息字段含义}
	\label{tab:场景空间信息字段含义}
	\begin{tabular}{lp{9cm}}
		\hline
		字段       & 含义                \\
		\hline
		weakness & 所属的缺陷CWE ID       \\
		\hline

		\makecell[l]{
		Program\_Span                \\
		程序跨度                         \\
		}
		         & \makecell[l]{
		0:单函数内部;                     \\
		1:上下文无关跨函数;                  \\
		2:上下文敏感跨函数;                  \\
		3x:跨文件,后缀 x 表示跨文件层数;         \\
		9:全程序(本维度上限)                 \\
		}
		\\
		\hline

		\makecell[l]{
		Semantic\_Complexity         \\
		语义建模复杂度                      \\
		}
		         & \makecell[l]{
		0:无语义;                       \\
		1:混淆正则匹配的语义,需要理解语法;          \\
		2:混淆if、while等\tc~语法,需要理解控制流; \\
		3:混淆控制流的语义,需要理解状态;           \\
		}
		\\
		\hline

		\makecell[l]{
		Path\_Complexity             \\
		路径结构复杂度                      \\
		}
		         & \makecell[l]{
		0:顺序结构;                      \\
		11:在if路径中;                   \\
		12:在else路径中;                 \\
		21:在while的路径中;               \\
		22:在for的路径中;                 \\
		23:在while的break路径中;          \\
		24:在for的break路径中;            \\
		3:在if和while/for复合的路径中;       \\
		9:在无限复杂的路径中(该维度的上限)          \\
		}
		\\
		\hline

		\makecell[l]{
		Path\_Depth                  \\
		路径深度与状态空间控制                  \\
		}
		         & \makecell[l]{
		0:不需要控制路径深度;                 \\
		1:if/else需要合并;               \\
		2:switch中多个case需要合并;         \\
		3:有无限增长的路径,工具规则需要能够兜底;       \\
		4x:在非平凡循环之外的if语句中,           \\
		后缀x表示非平凡循环层数,                \\
		需要有一定的展开能力;                  \\
		5:在大循环的末端,                   \\
		需要使用widen技术划分循环;             \\
		}
		\\
		\hline

		\makecell[l]{
		Reachability                 \\
		有无不可达路径                      \\
		}
		         & \makecell[l]{
		0:无不可达路径;                    \\
		1:有不可达路径;                    \\
		}
		\\
		\hline
	\end{tabular}
\end{table}

举例来说,
对于表达式\ref{eq:场景1}
描述的一个空间信息,

\begin{equation}
	\mathcal{p} = \langle
	D = 476,
	P = 1,
	S = 3,
	C = 12,
	L = 45,
	R = 0
	\rangle
	\label{eq:场景1}
\end{equation}

它是比较复杂的场景,
除了维度R都是非平凡的,
其含义是:
一个缺陷\texttt{CWE-476:NULL\_Pointer\_Dereference},
发生在如下的条件中:
数据的源和触发发生在上下文无关的跨函数之间;
触发条件发生在状态某种特定状态下;
发生在else分支中;
发生在非平凡循环外的分支中,
循环深度为5;
没有不可达路径。
具体代码见附录\ref{lst:场景1实际代码}


\subsection{自然语言描述与结构化信息的映射}
todo
场景需要被检索,
但是代码检索是另一个比较广泛的课题,
不是本文的重点。
本问需要的是通过自然语言检索到相关的代码片段,
生成功能分为两部分,
正向梳理和被动补充,
这两个方向的源头都是被描述成自然语言的代码问题。
下面分别举例正向梳理和被动补充时复杂空间信息,


从这个场景空间信息到场景的检索是不直接的,
需要将场景描述为自然语言,
以便场景空间信息和代码内容之间互相理解。




\subsection{评估结果与覆盖关系的存储方式}

\section{场景生成机制实现}
\section{场景去重、分类与管理实现}
\section{评估流程自动化实现}
\section{系统可扩展性与替换性分析}
% 评估系统实现与关键技术(只讲实现)
%   场景生成机制实现(RAG + 责任链)
%   场景去重、分类与管理实现
%   评估流程自动化实现
%   系统可扩展性与替换性分析
\chapter{实验与分析}
\label{ch:实验与分析}

\section{实验目的和实验设计}

本章围绕本文所提出的面向静态分析工具覆盖能力评估的系统,
设计并开展一系列实验,
旨在验证系统在场景生成效率与覆盖能力评估有效性两个核心方面的实际效果。
实验设计紧密对应前文提出的系统设计目标与评估维度,
在真实工具与真实规则条件下给出可解释的实验结论。

\subsection{实验目的}
具体而言,
本章实验主要包括以下两个目标:

\begin{enumerate}
    \item \textbf{验证基于管理模块与知识库约束的场景生成机制在效率与质量上的有效性}。
          具体来说,
          相比于传统依赖LLM直接生成测试场景加上人工执行校验和审核的流程,
          验证在此基础上进行检索、
          校验的场景生成是否能够在保证语义正确性的前提下,
          提高生成场景的人工审核通过率和效率。
          也就是实验系统是否能够在工程实践中降低场景构建与维护的成本;
    \item \textbf{验证所构造的场景空间在静态分析工具覆盖能力评估中的表达能力与解释能力}。
          具体来说,
          基于本文提出的覆盖能力评估维度体系,
          对一个工具在不同维度下的能力边界进行系统性刻画,
          并将实验结果与以往对该工具的经验性认识进行对比,
          分析该评估方式在揭示工具能力盲区与能力分布方面的有效性。
          也就是实验所构建的场景空间是否能够作为一种有效的载体,
          用于表达和评估静态分析工具的覆盖能力边界。
\end{enumerate}

\subsection{实验设计}

为保证实验结果的可比性与可解释性,
整体实验流程如图\ref{fig:实验流程图}所示。

\begin{figure}[h]
    \centering
    \includegraphics[width=\textwidth]{实验流程图.png}
    \caption{实验流程图}
    \label{fig:实验流程图}
\end{figure}

前置准备主要是为两个实验能够顺利进行而进行的启动工作,
主要是根据JTS丰富场景知识库、
构建规则映射和补充缺失场景,
这里本文不再赘述。

\subsubsection{实验一设计}

本实验设计如表\ref{tab:实验一设计}所示。

\begin{table}[h]
    \centering
    \caption{实验一设计}
    \label{tab:实验一设计}
    \begin{tabular}{|l|l|}
        \hline
        \textbf{实验目的} & 实验场景工厂在生成质量和效率上是否有优势    \\ \hline
        \textbf{实验对比} & 纯LLM生成场景                \\ \hline
        \textbf{实验对象} & 20组不同缺陷下的场景描述和需求        \\ \hline
        \textbf{实验指标} & 人工审核通过率、人工修改一行以内通过率、总耗时 \\ \hline
    \end{tabular}
\end{table}

实验目的
场景生成效率实验
该实验围绕“生成质量”与“人工校验成本”展开。
实验中分别采用:
基于本文系统的场景生成方式(结合知识库与生成约束);
不引入场景管理与约束条件的纯大语言模型生成方式;
对同一类规则需求进行场景生成。
生成结果统一交由人工进行语义与工程可用性审核,
并以人工审核通过率作为主要评价指标,
从而对比不同生成方式在实际工程使用中的效率差异。

\subsubsection{实验二设计}

构造的场景空间对 clang-tidy 工具评估的有效性实验
该实验以 clang-tidy 为评估对象,
选取其具有代表性的规则集合,
基于本文提出的覆盖能力评估维度,
对其在不同维度下的支持情况进行实验分析。
实验结果不以单一数值指标为目标,
而是通过场景触发结果与覆盖分布,
刻画 clang-tidy 在不同能力维度上的优势与局限,
并与传统基于文档或经验的工具认知进行对照分析。

\section{实验环境}

\subsection{实验环境}

为了保证场景知识库服务的可重现性、
环境隔离性与易运维性,
本文采用Docker容器化方式在本地/服务器上部署Elasticsearch(以下简称ES)与Kibana。
使用Docker的主要理由包括:
消除宿主环境依赖差异,
保证不同实验/评测环境的一致性;
便于版本固定与回滚,
利于科研复现;
简化运维与资源隔离,
便于在CI或实验平台中自动化启动与销毁实例;
通过卷(volume)实现数据持久化,
便于长期保存索引与检索结果,
也就是容器和数据的隔离。
Docker技术使用docker-compose部署多个容器,
本文使用的docker-compose部署示例如配置\ref{lst:docker-compose}所示。

\subsection{实验前置准备}

前置任务:
使用管理模块对JTS中的测例进行分类,
让知识库有基础的可参考知识,

\section{实验一:基于场景知识库的场景生成质量和效率实验}

做生成、校验实验,
指标是人工审核通过率,
对比是单纯的LLM生成。

\section{实验二:基于场景空间对clang-tidy工具评估的有效性实验}

以clang-tidy为实验对象,
研究它在我的评估维度下的能力。
对比是以往对该工具的认识。
主要是能力边界的认识。

\section{实验分析与讨论}

哪些维度最容易被覆盖?哪些最难?
系统在哪些类型规则上效果最好?
对路径敏感 / 约束复杂的规则,系统的局限在哪里?

重点展示:
覆盖维度数量
覆盖关系数量
场景规模变化
人力成本变化
管理复杂度变化
% 实验与分析
%   实验设计与评估目标
%   不同工具规则的表现分析
%   与传统评估方式的对比
%   威胁分析与讨论
\include{chapters/citations} % 引用


% % !TeX root = ../main.tex

\chapter{静态分析工具}

这里我们首先介绍静态分析工具的定义和特点

\section{静态分析工具定义和特点}

静态分析工具有其独特的特点如表~\ref{tab:feature_of_sa}。

\begin{table}
	\centering
	\bicaption{静态分析工具定义和特点}{english title}
	\label{tab:feature_of_sa}
	\begin{tabular}{cl}
		\toprule
		类型   & 内容  \\
		\midrule
		运行特点 & 特点1 \\
		检查特点 & 特点2 \\
		\bottomrule
	\end{tabular}
\end{table}

\section{}
% \include{chapters/floats}
% \include{chapters/math}

\bibliography{bib/ustc}  % 参考文献使用 BibTeX 编译
% \printbibliography       % 参考文献使用 BibLaTeX 编译

\appendix
% % !TeX root = ../main.tex

\chapter{补充材料}

\begin{lstlisting}[caption={场景1实际代码}, label={lst:场景1实际代码}]

void func(int cond){
	int x = 0;
	while (cond){
		x++;
	}
	if (x > 5){
		bug(x);
	}
}

\end{lstlisting}

\backmatter
% !TeX root = ../main.tex

\begin{acknowledgements}

感谢老师们!

\end{acknowledgements}

% !TeX root = ../main.tex

\begin{achievements}

	\begin{theachievements}[已发表论文]
		\item xx. 中国科学技术大学研究生学位论文撰写规范. 中国科学技术大学学报,2024,1.
		\item ……
	\end{theachievements}

\end{achievements}


\end{document}
